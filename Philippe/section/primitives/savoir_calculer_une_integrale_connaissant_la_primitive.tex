        	\begin{question}{60}{Primitives}{1}{/}
				Soit $f$ une fonctions définies sur $\mathbb{R}$ par $f(x)=a$ avec $a$ une constante. Sa primitive par rapport à $x$ est une fonction $g$ définie sur $\mathbb{R}$ par $g(x)=ax$. Que vaux l'intégrale de $1$ à $10$ de la fonction $f$ définie sur $\mathbb{R}$ par $f(x)=3$?
            \end{question}
            \begin{reponses}
            	\item[false] $27x$
            	\item[false] $3x$
                \item[false] 30
                \item[true] 27
            \end{reponses}
			%%%%%%%%%%%%%%%%%%%%%%%%%%%%%%%%%%%%%
            \begin{question}{60}{Primitives}{2}{/}
                Soit $f$ une fonctions définies sur $\mathbb{R}$ par $f(x)=x^n$. Sa primitive par rapport à $x$ est une fonction $g$ définie sur $\mathbb{R}$ par $g(x)=\frac{1}{n+1}x^{n+1}$. Que vaut $\int_{-2}^4 x^2 dx$?
            \end{question}
            \begin{reponses}
                \item[false] 12
                \item[false] 18
                \item[false] 4
                \item[true] 24
            \end{reponses}
            %%%%%%%%%%%%%%%%%%%%%%%%%%%%%%%%%%%%%
        	\begin{question}{60}{Primitives}{3}{/}
				Une primitive de $\ln$ est $x\mapsto x\ln(x)-x$. Que vaut $\int_{1}^e \ln(x)dx$?
            \end{question}
            \begin{reponses}
            	\item[false] e
            	\item[false] 0
                \item[true] 1
                \item[false] -1
            \end{reponses}
			%%%%%%%%%%%%%%%%%%%%%%%%%%%%%%%%%%%%%
            \begin{question}{60}{Primitives}{3}{/}
                Le travail d'une force est l'intégrale de la projection du travail sur le chemin parcouru. En termes mathématiques: $W=\int_{x_i}^{x_f}\vec{F}.\vec{dl}$. Un jouet avance de $x_i=0\;\si{\meter}$ à $x_f=2\;\si{\meter}$ le long de l'axe $x$, poussé par une force dirigée selon l'axe $x$ dont l'intensité varie dans l'espace telle que $\vec{F}(x) = (1+\frac{3}{2}x^2)\hat{u}_x$. Le produit scalaire $\vec{F}\cdot\vec{dl}$ devient alors $F\cdot dx$.
                Que vaut le travail de cette force?
            \end{question}
            \begin{reponses}
                \item[true] \SI{6}{J}.
                \item[false] \SI{4}{J}.
                \item[false] \SI{8}{J}.
                \item[false] \SI{2}{J}.
            \end{reponses}
            %%%%%%%%%%%%%%%%%%%%%%%%%%%%%%%%%%%%%
