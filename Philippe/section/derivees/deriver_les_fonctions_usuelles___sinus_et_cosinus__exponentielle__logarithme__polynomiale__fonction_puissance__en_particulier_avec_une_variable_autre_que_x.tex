        	\begin{question}{22}{Dérivées}{1}{/}
				Quelle est la dérivée par rapport à $x$ de la fonction $f$ définie sur $\mathbb{R}$ par $f(x)=x$?
            \end{question}
            \begin{reponses}
            	\item[false] $x^2$
            	\item[false] $2x$
                \item[false] 0
                \item[true] 1
            \end{reponses}
			%%%%%%%%%%%%%%%%%%%%%%%%%%%%%%%%%%%%%
        	\begin{question}{22}{Dérivées}{1}{/}
				Quelle est la dérivée par rapport à $x$ de la fonction $f$ définie sur $\mathbb{R}$ par $f(x)=x^2$?
            \end{question}
            \begin{reponses}
            	\item[false] $x$
            	\item[true] $2x$
                \item[false] 0
                \item[false] 1
            \end{reponses}
			%%%%%%%%%%%%%%%%%%%%%%%%%%%%%%%%%%%%%
        	\begin{question}{22}{Dérivées}{1}{/}
				Quelle est la dérivée par rapport à $x$ de la fonction $f$ définie sur $\mathbb{R}$ par $f(x)=\cos(x)$?
            \end{question}
            \begin{reponses}
            	\item[false] $\sin(x)$
            	\item[false] $-\cos(x)$
                \item[true] $-\sin(x)$
                \item[false] 1
            \end{reponses}
			%%%%%%%%%%%%%%%%%%%%%%%%%%%%%%%%%%%%%
        	\begin{question}{22}{Dérivées}{1}{/}
				Quelle est la dérivée par rapport à $x$ de la fonction $f$ définie sur $\mathbb{R}$ par $f(x)=\sin(x)$?
            \end{question}
            \begin{reponses}
            	\item[false] $-\cos(x)$
            	\item[true] $\cos(x)$
                \item[false] $-\sin(x)$
                \item[false] 1
            \end{reponses}
			%%%%%%%%%%%%%%%%%%%%%%%%%%%%%%%%%%%%%
        	\begin{question}{22}{Dérivées}{1}{/}
				Quelle est la dérivée par rapport à $x$ de la fonction $f$ définie sur $\mathbb{R}$ par $f(x)=e^x$?
            \end{question}
            \begin{reponses}
            	\item[true] $e^x$
            	\item[false] $e$
                \item[false] $xe^{x-1}$
                \item[false] 0
            \end{reponses}
			%%%%%%%%%%%%%%%%%%%%%%%%%%%%%%%%%%%%%
        	\begin{question}{22}{Dérivées}{2}{/}
				Quelle est la dérivée par rapport à $x$ de la fonction $f$ définie sur $\mathbb{R}$ par $f(x)=a\times g(x)$ avec $a$ une constante et $g$ une fonction définie sur $\mathbb{R}$?
            \end{question}
            \begin{reponses}
            	\item[false] $0$
            	\item[false] $\frac{dg}{dx}$
                \item[false] $\frac{dg}{d(a\times x)}$
                \item[true] $a\times \frac{dg}{dx}$
            \end{reponses}
			%%%%%%%%%%%%%%%%%%%%%%%%%%%%%%%%%%%%%
        	\begin{question}{22}{Dérivées}{2}{/}
				Quelle est la dérivée par rapport à $x$ de la fonction $f$ définie sur $\mathbb{R}$ par $f(x)=x^n$?
            \end{question}
            \begin{reponses}
            	\item[false] $nx^n$
            	\item[true] $nx^{n-1}$
                \item[false] $x^{n+1}$
                \item[false] $nx^{n+1}$
            \end{reponses}
			%%%%%%%%%%%%%%%%%%%%%%%%%%%%%%%%%%%%%
        	\begin{question}{22}{Dérivées}{2}{/}
				Quelle est la dérivée de par rapport à $x$ de la fonction $f$ définie sur $\mathbb{R}$ par $f(x)= g(a\times x)$ avec $a$ une constante et $g$ une fonction définie sur $\mathbb{R}$?
            \end{question}
            \begin{reponses}
            	\item[false] $0$
            	\item[false] $\frac{dg}{d(a\times x)}$
                \item[true] $a\times \frac{dg}{d(a\times x)}$
                \item[false] $a\frac{dg}{dx}$
            \end{reponses}
			%%%%%%%%%%%%%%%%%%%%%%%%%%%%%%%%%%%%%
            \begin{question}{22}{Dérivées}{2}{/}
                Que vaut la dérivée par rapport à $banane$ de la fonction qui à $banane$ associe $3e^{banane/\pi}$?
            \end{question}
            \begin{reponses}
                \item[true] $\frac{3}{\pi}e^{banane/\pi}$
                \item[false] $\frac{3\cdot banane}{\pi}e^{banane/\pi}$
                \item[false] $\frac{3}{\pi}\ln(banane/\pi)$
                \item[false] $\frac{3\cdot banane}{\pi}\ln(banane/\pi)$
            \end{reponses}
            %%%%%%%%%%%%%%%%%%%%%%%%%%%%%%%%%%%%%
            \begin{question}{22}{Dérivées}{2}{/}
                La capacité thermique isochore d'un corps est la variation, à volume $V$ constant, de l'énergie interne $U$ du corps par rapport à la température $T$ ($C_V=\frac{\partial U}{\partial T}$). Quelle est l'expression de la capacité thermique isochore si $U=\frac{3}{2}\cdot nRT$ avec $n$ et $R$ des constantes?
            \end{question}
            \begin{reponses}
                \item[false] $\frac{3}{2}\cdot nR+1$
                \item[false] $\frac{2}{3}\cdot nR$
                \item[true] $\frac{3}{2}\cdot nR$
                \item[false] $\frac{2}{3}\cdot nR+1$
            \end{reponses}
            %%%%%%%%%%%%%%%%%%%%%%%%%%%%%%%%%%%%%
            \begin{question}{22}{Dérivées}{3}{/}
                Que vaut la dérivée par rapport à $chien$ de la fonction qui à $chien$ associe $(chat^3\times chien)/dromadaire$?
            \end{question}
            \begin{reponses}
                \item[false] $(3 \times chat^2 \times chien)/dromadaire$
                \item[true] $(chat^3)/dromadaire$
                \item[false] $-(chat^3\times chien)/dromadaire^2$
                \item[false] $-(3 \times chat^2 \times chien)/dromadaire^2$
            \end{reponses}
            %%%%%%%%%%%%%%%%%%%%%%%%%%%%%%%%%%%%%
            \begin{question}{22}{Dérivées}{3}{/}
                Le nombre total d'atomes radioactifs $N(t)$ présents à un temps $t$ dans un échantillon contenant $N_0$ atomes à un temps $t_0=0$ est égal à $N(t)=N_0e^{-t/\tau}$, avec $\tau$ une constante. L'activité étant définie comme le nombre de désintégrations par secondes (\textit{i.e.} le négatif de la dérivée par rapport au temps de $N(t)$), quelle est son expression?
            \end{question}
            \begin{reponses}
                \item[false] $-\frac{N_0}{\tau}e^{-t/\tau}$
                \item[false] $-N_0-\frac{N_0}{\tau}e^{-t/\tau}$
                \item[true] $\frac{N_0}{\tau}e^{-t/\tau}$
                \item[false] $N_0+\frac{N_0}{\tau}e^{-t/\tau}$
            \end{reponses}
            %%%%%%%%%%%%%%%%%%%%%%%%%%%%%%%%%%%%%
