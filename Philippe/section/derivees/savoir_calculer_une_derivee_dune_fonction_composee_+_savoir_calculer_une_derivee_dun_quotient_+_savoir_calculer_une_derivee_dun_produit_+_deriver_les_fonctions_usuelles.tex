        	\begin{question}{/}{Dérivées}{1}{22 et 23 et 24 et 25}
				Soit $f$ une fonction définie sur $\mathbb{R}$ par $f(x)=e^{3x}$. Quelle est sa dérivée par rapport à $x$?
            \end{question}
            \begin{reponses}
            	\item[true] $3e^{3x}$
            	\item[false] $e^{3x}$
                \item[false] $3e$
                \item[false] $3xe^{3x}$
            \end{reponses}
			%%%%%%%%%%%%%%%%%%%%%%%%%%%%%%%%%%%%%
            \begin{question}{/}{Dérivées}{2}{22 et 23 et 24 et 25}
                Soit $z$ une fonction définie sur $\mathbb{R}$ par de $z(t)=-\sin(\cos(t))$. Que vaut sa dérivée par rapport à $t$?
            \end{question}
            \begin{reponses}
                \item[false] $\cos(t)\sin(\cos(t))-\sin(\cos(t))$
                \item[false] $\cos(t)\sin(\cos(t))$
                \item[true] $\sin(t)\cos(\cos(t))$
                \item[false] $\sin(t)\cos(\cos(t))-\sin(\sin(t))$
            \end{reponses}
            %%%%%%%%%%%%%%%%%%%%%%%%%%%%%%%%%%%%%
            \begin{question}{/}{Dérivées}{2}{22 et 23 et 24 et 25}
                Nous considérons un objet en mouvement circulaire accéléré. L'angle entre le vecteur position et l'axe des $x$ est $\theta(t)=\alpha t^2$ avec $\alpha$ une constante. La composante sur $x$ de la position est donnée par \mbox{$A\cos(\theta(t))=A\cos(\alpha t^2)$}, avec $A$ une constante. Quelle est l'expression de la dérivée de la composante sur $x$ par rapport au temps?
            \end{question}
            \begin{reponses}
                \item[false] $-2\alpha A\sin(\alpha t^2)$
                \item[false] $-A\sin(2\alpha t)$
                \item[false] $-2A\alpha\sin(t)$
                \item[true] $-2\alpha At\sin(\alpha t^2)$
            \end{reponses}
            %%%%%%%%%%%%%%%%%%%%%%%%%%%%%%%%%%%%%
            \begin{question}{/}{Dérivées}{3}{22 et 23 et 24 et 25}
                Soit $f$ une fonction définie sur $\mathbb{R}^2$ par $f(x,y)=\frac{x}{1-y^2}$. Que vaut sa dérivée par rapport à $y$?
            \end{question}
            \begin{reponses}
                \item[true] $\frac{2xy}{(1-y^2)^2}$
                \item[false] $\frac{-2x}{(1-y^2)^2}$
                \item[false] $\frac{2y}{(1-y^2)^2}$
                \item[false] $\frac{-2y}{(1-y^2)^2}$
            \end{reponses}
            %%%%%%%%%%%%%%%%%%%%%%%%%%%%%%%%%%%%%
