            \begin{question}{20}{Dérivées}{2}{19}
                Une fonction $f$ est strictement croissante sur un certain intervalle. Une autre fonction $g$ est constante sur le même intervalle. Quelle affirmation peut être juste?
            \end{question}
            \begin{reponses}
                \item[false] $f$ est la dérivée de $g$.
                \item[true] $g$ est la dérivée de $f$.
                \item[false] Les deux fonctions ont une primitive commune.
                \item[false] Les deux fonctions ont la même dérivée.
            \end{reponses}
            %%%%%%%%%%%%%%%%%%%%%%%%%%%%%%%%%%%%%
            \begin{question}{20}{Vecteurs}{2}{19}
                Quelles affirmations sont vraies?
                \begin{center}
                	\includegraphics[width=0.5\textwidth]{Philippe/Figures_Philippe/d_riv_es_2_6.png}
                \end{center}
            \end{question}
            \begin{reponses}
                \item[true] La courbe en pointillée est la dérivée de la courbe en trait plein
                \item[false] La courbe en traits pl est la dérivée de la courbe en trait plein
                \item[false] Si la courbe en pointillés est une vitesse en fonction du temps, alors la courbe en traits pleins est une accélération en fonction du temps.
                \item[true] Si la courbe en pointillés est une vitesse en fonction du temps, alors la courbe en traits pleins est une position en fonction du temps.
            \end{reponses}
            %%%%%%%%%%%%%%%%%%%%%%%%%%%%%%%%%%%%%
            \begin{question}{20}{Dérivées}{3}{19}
                La vitesse d'un objet est croissante dans le temps. Quelles affirmations peuvent être vraies à propos de sa position par rapport à un point fixe? (plusieurs réponses possibles)
            \end{question}
            \begin{reponses}
                \item[true] La position est positive.
                \item[true] La position est décroissante.
                \item[true] La position est croissante.
                \item[true] La position est négative.
            \end{reponses}
            %%%%%%%%%%%%%%%%%%%%%%%%%%%%%%%%%%%%%
