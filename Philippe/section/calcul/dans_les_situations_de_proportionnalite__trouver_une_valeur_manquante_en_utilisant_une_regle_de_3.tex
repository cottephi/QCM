            \begin{question}{1215}{Calcul}{1}{1219}
                On donne $8\times x=4$. Que vaut $x$?
            \end{question}
            \begin{reponses}
                \item[true] 0,5
                \item[false] 2
                \item[false] 12
                \item[false] 4
            \end{reponses}
            %%%%%%%%%%%%%%%%%%%%%%%%%%%%%%%%%%%%%
        	\begin{question}{1215}{Calcul}{2}{1219}
				On donne $\sqrt{2} \times \frac{x}{3}=1$. Que vaut $x$?
            \end{question}
            \begin{reponses}
            	\item[false] $\frac{\sqrt{2}}{3}$
            	\item[false] $1-\frac{3}{\sqrt{2}}$
                \item[false] $1-\frac{\sqrt{2}}{3}$
                \item[true] $\frac{3}{\sqrt{2}}$
            \end{reponses}
			%%%%%%%%%%%%%%%%%%%%%%%%%%%%%%%%%%%%%
            \begin{question}{1215}{Calcul}{2}{1219}
                La force de rappel d'un ressort est égal à $-k\Delta x$, avec $k$ la constante de rappel et $\Delta x$ la position de l'extrémité du ressort par rapport à la position d'équilibre. On mesure une force de \SI{-4}{\newton} et un écart à l'équilibre de $\Delta x=$\SI{0,05}{\meter}. Quelle est la valeur de $k$?
            \end{question}
            \begin{reponses}
                \item[false] \SI{0,0125}{\newton\per\meter}
                \item[true] \SI{80}{\newton\per\meter}
                \item[false] \SI{0,2}{\newton\per\meter}
                \item[false] \SI{10}{\newton\per\meter}
            \end{reponses}
            %%%%%%%%%%%%%%%%%%%%%%%%%%%%%%%%%%%%%
