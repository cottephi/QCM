        	\begin{question}{1216}{Calcul}{1}{1215}
				Que vaut $x$ pourcent de $y$?
            \end{question}
            \begin{reponses}
            	\item[false] $y\times x$
            	\item[true] $y\times \frac{x}{100}$
                \item[false] $\frac{y}{100x}$
                \item[false] $100y/x$
            \end{reponses}
			%%%%%%%%%%%%%%%%%%%%%%%%%%%%%%%%%%%%%
        	\begin{question}{1216}{Calcul}{1}{1215}
				Que vaut $y$ augmenté de $x$ pourcent?
            \end{question}
            \begin{reponses}
            	\item[false] $y\times (1+x)$
            	\item[false] $y\times \frac{x}{100}$
                \item[true] $y\times (1+\frac{x}{100})$
                \item[false] $101y/x$
            \end{reponses}
			%%%%%%%%%%%%%%%%%%%%%%%%%%%%%%%%%%%%%
            \begin{question}{1216}{Calcul}{2}{1215}
                Un article coûte, hors soldes, 60 euros. Son coût diminue de \SI{30}{\percent} durant les soldes. Combien coûte-t-il durant les soldes?
            \end{question}
            \begin{reponses}
                \item[false] 20 euros
                \item[false] 18 euros
                \item[true] 42 euros
                \item[false] 40 euros
            \end{reponses}
            %%%%%%%%%%%%%%%%%%%%%%%%%%%%%%%%%%%%%
            \begin{question}{1216}{Calcul}{2}{1215}
                On mesure l'activité d'une source radioactive et on trouve \SI{120}{\becquerel}. On mesure la même source un jour plus tard, et on trouve qu'elle a diminué de \SI{60}{\percent}. Que vaut cette nouvelle activité?
            \end{question}
            \begin{reponses}
                \item[false] \SI{72}{\becquerel}
                \item[true] \SI{48}{\becquerel}
                \item[false] \SI{20}{\becquerel}
                \item[false] \SI{100}{\becquerel}
            \end{reponses}
            %%%%%%%%%%%%%%%%%%%%%%%%%%%%%%%%%%%%%
