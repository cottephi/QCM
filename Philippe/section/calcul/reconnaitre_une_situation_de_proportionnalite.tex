            \begin{question}{1219}{Calcul}{1}{/}
            	6 patates permettent de nourrir 3 personnes pendant une soirée raclette. Avec 12 patates, combien de personnes est-il possible de nourrir?
            \end{question}
            \begin{reponses}
            	\item[false] 7
            	\item[true] 6
                \item[false] 5
                \item[false] 4
            \end{reponses}
			%%%%%%%%%%%%%%%%%%%%%%%%%%%%%%%%%%%%%
            \begin{question}{1219}{Calcul}{2}{/}
            	6 patates permettent de nourrir 3 personnes pendant une soirée raclette. Combien de personnes peuvent nourrir $n$ patates?
            \end{question}
            \begin{reponses}
            	\item[false] $3n$
            	\item[false] $2n$
                \item[false] $n$
                \item[true] $n/2$
            \end{reponses}
			%%%%%%%%%%%%%%%%%%%%%%%%%%%%%%%%%%%%%
            \begin{question}{1219}{Calcul}{2}{/}
            	Un objet est lâché sans vitesse initiale et n'est soumis qu'à son seul poids. Sa vitesse s'exprime alors comme $v(t)=gt$ avec $g$ une constante. Que vaut sa vitesse à $t=10\;\si{\second}$ sachant qu'elle valait \SI{49,05}{\meter\per\second} a $t=5\;\si{\second}$?
            \end{question}
            \begin{reponses}
            	\item[false] \SI{490,5}{\meter\per\second}
            	\item[false] \SI{981}{\meter\per\second}
                \item[false] \SI{49,05}{\meter\per\second}
                \item[true] \SI{98,1}{\meter\per\second}
            \end{reponses}
			%%%%%%%%%%%%%%%%%%%%%%%%%%%%%%%%%%%%%
