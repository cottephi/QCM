            \begin{question}{/}{Calcul}{2}{1210 et 1211}
                Que vaut $(\num{0.4} \times 120)/\num{0.3}$?
            \end{question}
            \begin{reponses}
                \item[true] \num{1.6e2}
                \item[false] \num{1.6e-2}
                \item[false] \num{1.6e1}
                \item[false] \num{1.6e-1}
            \end{reponses}
            %%%%%%%%%%%%%%%%%%%%%%%%%%%%%%%%%%%%%
            \begin{question}{/}{Calcul}{2}{1210 et 1211}
                On étudie un gaz parfait de volume $V=\SI{0.02}{\meter\cubed}$, de température $T=\SI{300}{\kelvin}$ et de quantité de matière $n=\SI{0.5}{\mole}$. Sachant que $P=nRT/V$ et que $R\simeq\SI{8}{\joule\per\mole\per\kelvin}$, la pression $P$ du gaz est:
            \end{question}
            \begin{reponses}
                \item[true] \SI{6e4}{\pascal}
                \item[false] \SI{6}{\pascal}
                \item[false] \SI{6e3}{\pascal}
                \item[false] \SI{6e1}{\pascal}
            \end{reponses}
            %%%%%%%%%%%%%%%%%%%%%%%%%%%%%%%%%%%%%
        	\begin{question}{/}{Calcul}{3}{1210 et 1211}
				Que vaut $\left(0,003 \times 40 \times 10^3 \times 5 \times 10^{-2} \right)/\left(0,6 \times 10^{1} \right) $?
            \end{question}
            \begin{reponses}
            	\item[false] 20
            	\item[false] 2
                \item[false] 10
                \item[true] 1
            \end{reponses}
			%%%%%%%%%%%%%%%%%%%%%%%%%%%%%%%%%%%%%
            \begin{question}{/}{Calcul}{3}{1210 et 1211}
                Le principe d'incertitude de Heisenberg énonce qu'il n'est jamais possible de connaître avec une infinie précision à la fois la position et l'impulsion d'un objet. L'équation qui décrit ce principe s'écrit $\sigma_x \times \sigma_p > \frac{\hbar}{2}$, où $\sigma_x$ et $\sigma_p$ sont les incertitudes sur la position et l'impulsion, respectivement, et $\hbar = \frac{h}{2\pi} \simeq \SI{1,7e-34}{\joule\second}$ est la constante de Planck réduite. Calculer l'incertitude minimum $\sigma_x$ dans le cas d'un humain dont l'impulsion est connue à $\pm \SI{5e-2}{\kilo\gram\meter\per\second}$ près.
            \end{question}
            \begin{reponses}
                \item[false] \SI{8,5e-35}{\meter}
                \item[true] \SI{1,7 e-33}{\meter}
                \item[false] \SI{8,5e-34}{\meter}
                \item[false] \SI{1,7e-32}{\meter}
            \end{reponses}
            %%%%%%%%%%%%%%%%%%%%%%%%%%%%%%%%%%%%%
