        	\begin{question}{1226}{Modélisation}{2}{non créé}
				Lors d'un examen, un étudiant doit calculer l'expression de la vitesse d'un objet. Laquelle de ces formules peut être correcte? ($g=\SI{9,81}{\meter\per\second^20}$, $t$ est un temps, $d$ une distance, $m$ une masse).
            \end{question}
            \begin{reponses}
            	\item[false] $\frac{gt}{d}$
            	\item[true] $gt$
                \item[false] $\frac{d}{gt}$
                \item[false] $gd$
            \end{reponses}
			%%%%%%%%%%%%%%%%%%%%%%%%%%%%%%%%%%%%%
            \begin{question}{1226}{Modélisation}{2}{non créé}
				Lors d'un examen, un étudiant doit calculer l'expression de l'énergie cinétique d'un objet. Une réponse est correcte, laquelle? ($g=\SI{9,81}{\meter\per\second^20}$, $t$ est un temps, $d$ une distance, $m$ une masse).
            \end{question}
            \begin{reponses}
                \item[false] $\frac{1}{2}mgt$
                \item[false] $\frac{1}{2}mgdt^2$
                \item[true] $\frac{1}{2}mg^2t^2$
                \item[false] $\frac{1}{2}mg^2d^2$
            \end{reponses}
			%%%%%%%%%%%%%%%%%%%%%%%%%%%%%%%%%%%%%
        	\begin{question}{1226}{Modélisation}{3}{non créé}
				A quelle grandeur physique peut correspondre $\gamma m v^2$ sachant que $m$ est une masse, $v$ une vitesse, et que $\gamma=\frac{1}{\sqrt{1-\frac{v^2}{c^2}}}$, avec $c$ la vitesse de la lumière?
            \end{question}
            \begin{reponses}
            	\item[false] une quantité de mouvement
            	\item[false] une force
                \item[false] une vitesse
                \item[true] une énergie
            \end{reponses}
			%%%%%%%%%%%%%%%%%%%%%%%%%%%%%%%%%%%%%
