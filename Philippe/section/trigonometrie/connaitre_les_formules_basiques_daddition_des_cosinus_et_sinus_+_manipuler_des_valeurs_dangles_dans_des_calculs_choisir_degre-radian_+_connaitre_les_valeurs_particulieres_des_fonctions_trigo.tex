            \begin{question}{/}{Trigonométrie}{2}{47 et 31}
                Que vaut $\sin(\frac{\pi}{2}+\frac{3\pi}{4})$? 
            \end{question}
            \begin{reponses}
                \item[true] $\sin = -\frac{\sqrt{2}}{2}$
                \item[false] $\sin = -1/2$
                \item[false] $\sin = \frac{\sqrt{2}}{2}$
                \item[false] $\sin = 1/2$
            \end{reponses}
            %%%%%%%%%%%%%%%%%%%%%%%%%%%%%%%%%%%%%
            \begin{question}{/}{Trigonométrie}{3}{47 et 31}
                En utilisant $\sin(\alpha+\beta)=\sin(\alpha)\cos(\beta) + \cos(\alpha)\sin(\beta)$, calculer $\sin(\frac{13\pi}{12})$.
            \end{question}
            \begin{reponses}
                \item[false] $-\frac{\sqrt{3}-1}{2\sqrt{3}}$
                \item[true] $-\frac{\sqrt{3}-1}{2\sqrt{2}}$
                \item[false] $\frac{\sqrt{3}+1}{2\sqrt{2}}$
                \item[false] $\frac{\sqrt{3}+1}{2\sqrt{3}}$
            \end{reponses}
            %%%%%%%%%%%%%%%%%%%%%%%%%%%%%%%%%%%%%
            \begin{question}{/}{Trigonométrie}{3}{47 et 31}
                On mesure l'angle d'un pendule à un temps $t_1$: $\theta_1=1,37$ radians. Au temps $t_2$, l'angle a augmenté de \num{1.47}~radians. Sachant que $\cos(\num{1.37})\simeq \num{0.2}$, $\sin(\num{1.37})\simeq \num{0.98}$, $\cos(\num{1.47})\simeq \num{0.1}$ et $\sin(\num{1.47})\simeq \num{0.99}$, calculez le cosinus de l'angle en $t_2$.
            \end{question}
            \begin{reponses}
                \item[false] $\cos(\theta_2) \simeq \num{0.296}$
                \item[true] $\cos(\theta_2) \simeq \num{-0.96}$
                \item[false] $\cos(\theta_2) \simeq \num{-1}$
                \item[false] $\cos(\theta_2) \simeq \num{1}$
            \end{reponses}
            %%%%%%%%%%%%%%%%%%%%%%%%%%%%%%%%%%%%%
