        	\begin{question}{60}{Primitives}{1}{/}
				La primitive de la fonction f définie sur $\RR$ par : $\forall x \in \RR, f(x)=a, a \in \RR$ est la fonction F définie sur $\RR$ par : $\forall x \in \RR, F(x)=ax, a \in \RR $. Que vaut l'intégrale $\int_{1}^{10} f(x) dx$ avec $\forall x \in \RR, f(x)=3$ ? 
            \end{question}
            \begin{reponses}
            	\item[false] $27x$
            	\item[false] $3x$
                \item[false] 30
                \item[true] 27
            \end{reponses}
			%%%%%%%%%%%%%%%%%%%%%%%%%%%%%%%%%%%%%
            \begin{question}{60}{Primitives}{2}{/}
                Soit $f$ une fonctions définies sur $\mathbb{R}$ par $f(x)=x^n$. Sa primitive par rapport à $x$ est une fonction $g$ définie sur $\mathbb{R}$ par $g(x)=\frac{1}{n+1}x^{n+1}$. Que vaut $\int_{-2}^4 x^2 dx$?
            \end{question}
            \begin{reponses}
                \item[false] 12
                \item[false] 18
                \item[false] 4
                \item[true] 24
            \end{reponses}
            %%%%%%%%%%%%%%%%%%%%%%%%%%%%%%%%%%%%%
