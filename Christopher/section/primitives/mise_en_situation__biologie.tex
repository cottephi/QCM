            \begin{question}{22}{Dérivées}{2}{/}
			 A l'instant t=0, on introduit $N_0$ bactéries dans un milieu de culture. On s'intéresse alors à l'évolution de la population de bactéries, dont le nombre à l'instant t (t > ou égal à 0). On connaît le taux d'accroissement des bactéries (la vitesse d'évolution) qui est défini par la fonction suivante $g(t) = N_0 \exp{10t}+3$. Pour connaître le nombre de bactérie à un instant $t$ il faut intégrer le taux d'accroissement par rapport au temps, c'est-à-dire qu'il faut calculer la primitive $G(t)$ définie par $G(t)= \int g(t')dt'$. Donner l'expression de $G(t)$.
            \end{question}
            \begin{reponses}
            	\item[false]   $\frac{N_0}{10}\exp{10t} $
            	\item[false]   $N_0\exp{10t} +3t $
                \item[true]   $\frac{N_0}{10}\exp{10t} +3t $
                \item[false]   $N_0\exp{10t} +3 $
            \end{reponses}
			%%%%%%%%%%%%%%%%%%%%%%%%%%%%%%%%%%%%%
