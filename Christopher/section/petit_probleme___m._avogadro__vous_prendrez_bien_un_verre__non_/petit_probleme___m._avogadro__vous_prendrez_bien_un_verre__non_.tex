            \begin{question}{NC}{César partie I }{1}{/}
				 Exprimer le lien entre le nombre d'entités chimiques $N$ contenues dans $n$ moles et le nombre d'Avogadro $N_a$
            \end{question}
            \begin{reponses}
            	\item[false] $\frac{n}{N_a} $
            	\item[false] $n^{N_a}$
                \item[true]  $n\times N_a $
                \item[false] $\frac{N_a}{n} $
            \end{reponses}
			%%%%%%%%%%%%%%%%%%%%%%%%%%%%%%%%%%%%%
			\begin{question}{NC}{César partie Ibis }{1}{/}
				 Exprimer le lien entre la masse volumique $\rho$ , le volume $V$ d'une solution et sa masse m.
            \end{question}
            \begin{reponses}
            	\item[true] $\rho = \frac{m}{V} $
            	\item[false] $\rho= m^{V}$
                \item[false]  $\rho = m\times V $
                \item[false] $\rho = \frac{V}{m} $
            \end{reponses}
			%%%%%%%%%%%%%%%%%%%%%%%%%%%%%%%%%%%%%
            \begin{question}{NC}{César partie II}{2}{/}
				  Combien y a-t-il de moles dans un litre d'eau? (info: 
				  $M(\ce{H}) = \SI{1.0}{\gram\per\mole}$;
				  $M(\ce{O}) = \SI{16.0}{\gram\per\mole}$;
				  $\rho_{\ce{H2O}} = \SI{1.0}{\gram\per\centi\meter\cubed}$)
            \end{question}
            \begin{reponses}
            	\item[false] \SI{62.5}{\mole}
            	\item[true]  \SI{55.5}{\mole}
                \item[false] \SI{0.625}{\mole}
                \item[false] \SI{0.555}{\mole}
            \end{reponses}
			%%%%%%%%%%%%%%%%%%%%%%%%%%%%%%%%%%%%%
            \begin{question}{NC}{César partie III}{2}{/}
				 \frquotes{Tu quoque mi fili!} aurait dit Jules César quand ce dernier fut assassiné par son fils Brutus. Imaginons que le dernier geste de César fut de boire un verre d'eau ($V=\SI{30}{\centi\liter}$), combien de molécules d'eau y aurait-t-il eu dans son verre ? (info:  $N_a = \SI{6.02e23}{\per\mole}$ )
            \end{question}
            \begin{reponses}
            	\item[true]    \SI{1.1e25}{molécules}
            	\item[false]   \SI{.71e20}{molécules}
                \item[false]   \SI{.11e20}{molécules}
                \item[false]   \SI{7.1e25}{molécules}
            \end{reponses}
			%%%%%%%%%%%%%%%%%%%%%%%%%%%%%%%%%%%%%
            \begin{question}{NC}{César partie IV}{2}{/}
				  Considérons toujours le verre d'eau de César. Supposons que grâce aux cycles de l’eau depuis \SI{2000}{ans}, toutes les molécules d’eau qu'il contenait aient été réparties aléatoirement dans le stock mondial d’eau.  Sachant qu'il y a sur Terre environ \num{1.4}~milliards de \si{\kilo\meter\cubed} d’eau, soit environ \num{5e46} molécules, quelle est la probabilité $p$ d'avoir une molécule d'eau dans votre verre de \SI{30}{\centi\liter} ayant appartenu au verre de César? 
            \end{question}
            \begin{reponses}
            	\item[false] $p = \num{1.4e-27}$
            	\item[true] $p = \num{2.2e-22}$
                \item[false] $p = \num{1.4e-21}$
                \item[false] $p = \num{2.2e-28}$
            \end{reponses}
			%%%%%%%%%%%%%%%%%%%%%%%%%%%%%%%%%%%%%
            \begin{question}{NC}{César partie V}{2}{/}
				  Connaissant la probabilité de tomber sur une molécule d'eau ayant appartenu à César dans votre verre, combien de molécules d'eau de César y a-t-il dans ledit verre de \SI{30}{\centi\liter}?  
            \end{question}
            \begin{reponses}
            	\item[true] Environ \num{2e3}.
            	\item[false]   Environ \num{2e5}.
                \item[false]   Environ \num{2e-1}.
                \item[false]   Environ \num{2e-3}.
            \end{reponses}
			%%%%%%%%%%%%%%%%%%%%%%%%%%%%%%%%%%%%%
