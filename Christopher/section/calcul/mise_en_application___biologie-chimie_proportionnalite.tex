            \begin{question}{1215}{dilution}{1}{}
                Lors d'une dilution, pour faire une solution fille de concentration $C_f$ et de volume $V_f$ à partir d'une solution mère concentration molaire $C_m$, on prélève un volume $V_m$ de la solution mère que l'on dilue. Quelle est relation reliant la concentration des solutions mère et fille au volume prélevé de la solution mère et au volume de la solution fille ?
            \end{question}
            \begin{reponses}
                \item[true] $C_f\cdot V_f = C_m\cdot V_m$
                \item[false] $C_f\cdot V_m = C_m\cdot V_f$
                \item[false] $C_f/V_f = C_m/V_m$
        	    \item[false] $V_f/C_f = V_m/C_m$
            \end{reponses}
            %%%%%%%%%%%%%%%%%%%%
            \begin{question}{1215}{Chimie}{1}{/}
    			 Soit $N$ le nombre d'entités chimiques contenu dans un échantillon. Que vaut $N$ en fonction de  $n$, le nombre de moles dans cet échantillon, et le nombre d'Avogadro $N_a$, qui correspond au facteur de conversion entre le gramme et l'unité de masse atomique. 
            \end{question}
            \begin{reponses}
            	\item[false] $ \frac{n}{N_a} $
            	\item[false] $ n^{N_a}$
                \item[true]  $ n.N_a $
                \item[false] $ \frac{N_a}{n} $
            \end{reponses}
			%%%%%%%%%%%%%%%%%%%%%%%%%%%%%%%%%%%%%
		    \begin{question}{1210}{calcul}{2}{/}
				  Combien y a-t-il de moles d'eau $n$ dans un litre d'eau ? (info : $M(H) =1,0\ g.mol^{-1}$ ;  $M(O) = 16,0\ g.mol^{-1}$)
            \end{question}
            \begin{reponses}
            	\item[false] $\SI{6.25e1}{\mol} $
            	\item[true]  $ \SI{5.55e1}{\mol} $
                \item[false] $ \SI{5.55e-1}{\mol} $
                \item[false] $ \SI{6.25e-1}{\mol} $
            \end{reponses}
			%%%%%%%%%%%%%%%%%%%%%%%%%%%%%%%%%%%%%
			\begin{question}{1210}{calcul}{3}{/}
				  Les bactéries se reproduisent par division binaire, phase de division de durée fixe que nous appellerons période. On note $P_n$ la population au bout de n périodes de reproduction, avec $n \in \NN $. On peut montrer que le nombre de bactéries au cycle n est défini par $P_n = 2^n$. Sachant que $\forall (a,x) \in \RR, ln(a^x) = x\ln{a}$, trouver à partir de quel cycle le nombre de bactéries dépasse $10^6$. 
            \end{question}
            \begin{reponses}
            	\item[false] 15
            	\item[false] 8
                \item[true]  20
                \item[false] 50
            \end{reponses}
			%%%%%%%%%%%%%%%%%%%%%%%%%%%%%%%%%%%%%
			\begin{question}{1210}{calcul}{2}{/}
				 A partir d’une solution mère de S on prépare une solution fille S' en prélevant un volume $V = \SI{350}{\micro\liter}$ de la solution mère et en rajoutant certain volume d’eau. Le volume $V'$ de la solution fille vaut $V' = \SI{100}{\milli\liter}$. Que vaut le facteur de dilution FD ? (c'est-à-dire : Combien de fois a-t-on dilué la solution mère ?) 
            \end{question}
            \begin{reponses}
            	\item[false]  $\SI{2.86e-3}{} $
            	\item[false]  $ \SI{3.50e-6}{}$
                \item[false]  $ \SI{2.86e2}{}$
                \item[true]   $ \SI{3.50e-3}{}$
            \end{reponses}
			%%%%%%%%%%%%%%%%%%%%%%%%%%%%%%%%%%%%%
             \begin{question}{}{dilution}{2}{1215,1209,1210,1211}
				 A partir d’une solution mère de $H_2SO_4$ « 1M »  (concentration $C_{mere} = \SI{1} {\mole\per\liter}$), vous devez préparer $\SI{500}{\milli\liter}$ de $H_2SO_4$ « 50mM » (concentration $C_{fille} = 50 \si{\milli\mole\per\liter}$). Quel volume de solution mère doit être prélevé?  (En pratique, ce volume sera rajouté à un volume d'eau pour avoir le volume de solution désiré).
            \end{question}
            \begin{reponses}
            	\item[true]  $\SI{25}{\milli\liter} $
            	\item[false]  $ \SI{25e-2}{\liter} $
                \item[false]   $ \SI{250}{\milli\liter} $
                \item[false]  $ \SI{0.25}{\milli\liter} $
            \end{reponses}
			%%%%%%%%%%%%%%%%%%%%%%%%%%%%%%%%%%%%%
        	\begin{question}{1210}{Grandeurs chimiques}{1}{/}
				On considère une solution de concentration molaire $C$ et de volume $V$. Exprimer $C$ en fonction de la quantité de moles $n$ dans la solution et de  $V$. 
            \end{question}
            \begin{reponses}
            	\item[true]  $\frac{n}{V}$
            	\item[false] $ n V$
                \item[false] $ \frac{V}{n}$
                \item[false] $ n^{V}$
            \end{reponses}
			%%%%%%%%%%%%%%%%%%%%%%%%%%%%%%%%%%%%%
            \begin{question}{1210}{Grandeurs chimiques}{2}{/}
				Soit une masse $m$ de sel que l'on dilue dans $\SI{5}{\deci\liter}$ d'eau. Sachant que la concentration massique $C_m$ de sel dissout dans l'eau est de $\SI{3e-2}{\gram\per\liter}$, quelle est la masse de sel initiale ?  
            \end{question}
            \begin{reponses}
            	\item[false] $\SI{15}{\gram}$
            	\item[false] $\SI{6e-2}{\gram}$
                \item[false] $\SI{1,6e-1}{\gram}$
                \item[true]  $\SI{15e-3}{\gram}$
            \end{reponses}
			%%%%%%%%%%%%%%%%%%%%%%%%%%%%%%%%%%%%%
            \begin{question}{1210}{Grandeurs chimiques}{2}{/}
				  Vous devez préparer \SI{150}{mL} de gel à \SI{1.2}{\percent} d’agarose. Un produit dosé à Y \% signifie qu'il a Y grammes de produit actif pour 100 ml. Quelle quantité $m$ d’agarose devez-vous peser ?
            \end{question}
            \begin{reponses}
            	\item[true]  \SI{1.8}{\gram}
            	\item[false]  \SI{8e-5}{\gram}
                \item[false]  \SI{0.3}{\milli\gram}
                \item[false]   \SI{2e-2}{\milli\gram}
            \end{reponses}
			%%%%%%%%%%%%%%%%%%%%%%%%%%%%%%%%%%%%%
			\begin{question}{1210}{proportionalité}{2}{/}
		        Les bactéries Escherichia Coli se reproduisent par division binaire, phase de division de durée fixe que nous appellerons période. On note $P_n$ la population au bout de n périodes de reproduction, avec $n \in \NN $. On peut montrer que le nombre de bactéries au cycle n est défini par $P_n = 2^n$. Pour 24 heures, ce qui représentent 72 périodes de reproduction, nous aurions une population de $2^{72} = 4,72.10^{21}$ bactéries. La masse d’une bactérie étant d’environ $\SI{7e-13}{\gram}$, quelle sera la masse obtenue au bout de 24 heures ? 
            \end{question}
            \begin{reponses}
            	\item[false]  $ \SI{1457}{\kilo\gram} $
            	\item[false]  $ \SI{2.45}{\giga\tonne} $
                \item[true]  $ \SI{1170}{\tonne} $
                \item[false]   $ \SI{5,97e24}{\kilo\gram} $
            \end{reponses}
			%%%%%%%%%%%%%%%%%%%%%%%%%%%%%%%%%%%%%
			\begin{question}{1210}{Calcul}{2}{/}
				 A partir d’une solution mère de S on prépare une solution fille S' en prélevant un volume $V = \SI{350}{\micro\liter}$ de la solution mère et en rajoutant un certain volume de solution tampon. Le volume $V'$ de la solution fille vaut $V' = \SI{100}{\milli\liter}$. Combien de fois a-t-on dilué la solution mère ? (i.e quel est le facteur de dilution FD ? ) 
            \end{question}
            \begin{reponses}
            	\item[false]  $FD = \SI{2.86e-3}{} $
            	\item[false]  $FD = \SI{3.50e-6}{}$
                \item[false]  $FD = \SI{2.86e2}{}$
                \item[true]   $FD = \SI{3.50e-3}{}$
            \end{reponses}
			%%%%%%%%%%%%%%%%%%%%%%%%%%%%%%%%%%%%%
			\begin{question}{1216,8,1209}{Calcul de concentration}{2}{}
				Les antibiotiques sont des molécules possédant la propriété de tuer des bactéries ou d'en limiter la propagation. La concentration dans le sang, en $\si{\milli\gram\per\liter}$, en fonction du temps d'un antibiotique injecté en une seule prise à un patient est défini par la fonction suivante $g(t) = \frac{4t}{t^{2}+1}$. t est le temps en heure. Au bout de combien d'heures, la concentration en antibiotique dans le sang vaut \SI{2}{\milli\gram\per\liter}  ?
            \end{question}
            \begin{reponses}
            	\item[true] 1 heure
            	\item[false]  2 heures
                \item[false]  3 heures
                \item[false] 4  heures
            \end{reponses}
			%%%%%%%%%%%%%%%%%%%%%%%%%%%%%%%%%%%%%
				\begin{question}{1216}{Calcul}{2}{1215}
                On étudie l'évolution d'une population de bactéries présentent dans une boite de Pétri. À $t=\SI{0}{\hour}$, on en compte \num{1500}. Au bout de \SI{2}{\hour}, leur nombre a été augmenté de \SI{120}{\percent}. Combien compte-t-on de bactéries à $t=\SI{2}{\hour}$?
            \end{question}
            \begin{reponses}
            	\item[false] \num{181500}
            	\item[true]  \num{3300}
                \item[false]  \num{1800}
                \item[false] \num{1262}
            \end{reponses}
			%%%%%%%%%%%%%%%%%%%%%%%%%%%%%%%%%%%%%
			\begin{question}{1216}{Calcul}{3}{1215}
                On étudie l'évolution d'une population de bactéries présentent dans une boite de Pétri. À $t=\SI{0}{\hour}$, on en compte \num{1000}. Au bout de \SI{2}{\hour}, on en compte \num{3000}. De combien à augmenté le nombre de bactéries? 
            \end{question}
            \begin{reponses}
            	\item[false] \SI{300}{\percent}
            	\item[true]  \SI{200}{\percent}
                \item[false] \SI{100}{\percent}
                \item[false] \SI{600}{\percent}
            \end{reponses}
			%%%%%%%%%%%%%%%%%%%%%%%%%%%%%%%%%%%%%
         \begin{question}{60,61,1214}{BIo}{3}{}
          On s'intéresse à l'évolution d'une population de bactéries en fonction du temps. Soit N(t) le nombre de bactérie au temps t défini par l'expression suivante : $N(t) = (x^2+3)^2 $. Calculer le nombre de bactéries observées entre le début de l'expérience (t=0h) et la fin de l'expérience (t=10h), c'est à dire calculer $\int^{10}_{0} N(t) \mathrm{d}t$.
        \end{question}
        \begin{reponses}
            \item[false] $12568$
            \item[true] $22090$
            \item[false]  $+\infty$
            \item[false] $-12568$
        \end{reponses}
        %%%%%%%%%%%%%%%%%%%%
            \begin{question}{8}{équations}{2}{/} 
            	Pour savoir si une expérience est compatible avec le modèle théorique utilisé, on utilise l'erreur relative $\delta = \frac{|theorie-experience|}{theorie}$. Sur une population de 300 mouches, on compte le nombre de mouches aux yeux rouges. Si on considère qu'un seul gène est à l'origine de la couleur des yeux, on estime à 3/4 le ratio entre le nombre de mouches rouges aux yeux rouges et le nombre total de mouches. Sachant que l'erreur relative calculée est de $6.6\%$, quel est le nombre de mouches aux yeux rouges comptés ? 
            \end{question}
            \begin{reponses}
            	\item[false]  150
            	\item[false]  50
                \item[false]  250
                \item[true]   210
            \end{reponses}
			%%%%%%%%%%%%%%%%%%%%%%%%%%%%%%%%%%%%%
