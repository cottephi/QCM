            \begin{question}{NC}{type de fonction}{2}{/} 
				Pour une enzyme michaélienne, la vitesse de synthèse d’un produit est définie selon  la relation: $v(S,K_m) = \frac{(V_{max} . S)}{(K_m+S)}$, avec $S$ la concentration en substrat, $K_m$ un paramètre de la cellule et $V_{max}$ la vitesse de synthèse maximale. On veut connaître l'influence de $K_m$ sur l'évolution de $v$. On étudie alors la dérivée partielle de $v$ par rapport à $K_m$ notée $\frac{\partial v}{\partial K_m}$. Donner son expression. 
            \end{question}
            \begin{reponses}
            	\item[false]  $\frac{\partial v}{\partial K_m} = \frac{V_{max}.S^2}{(K_m+S)^2}$
            	\item[false]   $\frac{\partial v}{\partial K_m} = \frac{-1}{(K_m+S)^2}$
                \item[true]  $\frac{\partial v}{\partial K_m} = \frac{-V_{max}.S^2}{(K_m+S)^2}$
                \item[false]  $\frac{\partial v}{\partial K_m} = \frac{-V_{max}}{(K_m+S)}$
            \end{reponses}
			%%%%%%%%%%%%%%%%%%%%%%%%%%%%%%%%%%%%%
				\begin{question}{NC}{dérivée}{2}{/} 
				La relation suivante montre l'évolution du nombre de cellules en fonction du temps : $N_{cellulles}(t)=\frac{10}{1-5.exp(-0.5t)}$. Exprimer la dérivée par rapport au temps de cette fonction.
            \end{question}
            \begin{reponses}
            	\item[true]   $\frac{-25.\exp(-0.5t)}{(1-5.\exp(-0.5t))^{2}}$ 
            	\item[false]  $\frac{10(1-5.\exp(-0.5t))-25.\exp(-0.5t)}{(1-5.\exp(-0.5t))^{2}}$
                \item[false]  ${10(1-5.\exp(-0.5t))-25.\exp(-ct)}$ 
                \item[false]   ${10(1-5.\exp(-0.5t))+25.\exp(-ct)}$
            \end{reponses}
			%%%%%%%%%%%%%%%%%%%%%%%%%%%%%%%%%%%%%
			\begin{question}{22}{Dérivées}{2}{/}
			  Lorsqu'un bactéricide a été introduit dans un milieu nutritif où une population de bactéries croissait, la population a continué à croître pendant un certains temps mais peu après elle s'est arrêtée de croître et a commencé à décliner. La taille de la population à l'instant t est $f(t) = 10^6 + 10^4 t - 10^3 t^2$. $f(t)$ représente le nombre de bactéries et t représente le temps en heures. Calculer le taux de croissance $\frac{\Delta individus}{\Delta t}$ au bout de 10h, c'est-à-dire la dérivée par rapport au temps $f'(t)$ à t=10h.
            \end{question}
            \begin{reponses}
            	\item[false] $10000$ individus/heure
            	\item[true]  $-10000$ individus/heure
                \item[false]   $0$ individus/heure
                \item[false] $5000$ individus par heure
                \item[false] $-5000$ individus par heure
            \end{reponses}
			%%%%%%%%%%%%%%%%%%%%%%%%%%%%%%%%%%%%%
			\begin{question}{22}{Dérivées}{2}{/}
			  Lorsqu'un bactéricide a été introduit dans un milieu nutritif où une population de bactéries croissait, la population a continué à croître pendant un certains temps mais peu après elle s'est arrêtée de croître et a commencé à décliner. La taille de la population à l'instant t est $f(t) = 10^6 + 10^4 t - 10^3 t^2$. $f(t)$ représente le nombre de bactéries et t représente le temps en heures. A quelle heure le taux de croissance des bactéries sera-t-il nul, c'est-à-dire a quel temps t $f'(t)=0$ ?
            \end{question}
            \begin{reponses}
            	\item[false] 0
            	\item[false] 10
                \item[false]  $+\infty$ 
                \item[true]  5
            \end{reponses}
			%%%%%%%%%%%%%%%%%%%%%%%%%%%%%%%%%%%%%
			\begin{question}{22}{Dérivées}{2}{/}
			 A l'instant t=0, on introduit $N_0$ bactéries dans un milieu de culture. On s'intéresse alors à l'évolution de la population de bactéries, dont le nombre à l'instant t (t > ou égal à 0) est défini par la fonction suivante $f(t) = N_0 \exp{10t}$. La dérivée par rapport au temps de $f$ correspond au taux d'accroissement du nombre de bactéries par unité de temps. Exprimer ce taux d'accroissement ?
            \end{question}
            \begin{reponses}
            	\item[true]   $10 N_0 \exp{10t} $
            	\item[false]  $N_0 e $
                \item[false]  $10 N_0 t \exp{10t-1} $
                \item[false]  $0 $
            \end{reponses}
			%%%%%%%%%%%%%%%%%%%%%%%%%%%%%%%%%%%%%
            \begin{question}{20}{Dérivées}{2}{19}
            La courbe ci-dessous représente l'évolution d'une population de bactérie en fonction du temps. Que peut-on dire de la dérivée de cette fonction à $t=\SI{1}{\hour}$?
            \begin{figure}[!h]
	          \begin{center}
              \begin{tikzpicture}
                  \begin{axis}[
                        title = { },
                        axis lines = left,
                        xlabel = $t (h)$,
                        %minor x tick num = 4,
                        ylabel = $N(t)$,
                        %ymin=0, ymax=40,
                        /pgf/number format/.cd,%3 lignes dessous, utiliser spacers français au lieu d'anglais.
                        use comma,
                        1000 sep={\,}
                      ]
                      %Below the red curve
                      \addplot [
                        domain=0:10,
                        samples=100,
                        color=red,
                        %/pgf/text mark = {+}, %changer le marqueur text
                        %mark=o,
                      ]
                      {4000*x/(x^2+1)};
                  \end{axis}
              \end{tikzpicture}
              \end{center}
              \end{figure}
        \end{question}
        \begin{reponses}
            	\item[false]  Elle est positive.
            	\item[true]  Elle est nulle.
                \item[false]  Elles est négative.
                \item[false]   Elle n'est pas définie.
            \end{reponses}
           %%%%%%%%%%%%%%%%%%%%%%%%%%%%%%%%%%%%%%%%%%%%%%%%%
			 \begin{question}{/}{Dérivées}{2}{22 et 23}
				La relation suivante montre l'évolution du nombre de cellules en fonction du temps: $N_{cellules}(t)=t^{2}\cdot\exp(5t)$. Quelle est la dérivée par rapport au temps de cette fonction?
            \end{question}
            \begin{reponses}
            	\item[true]   $2t\exp(5t)+10t^{2}\exp(5t)$ 
            	\item[false]  $2t\exp(5t)-10t\exp(5t) $ 
                \item[false]  $2t^{2}+5t\exp(5t) $ 
                \item[false]  $2t\exp(5t)+10t^{2}\exp(5t) $ 
            \end{reponses}
			%%%%%%%%%%%%%%%%%%%%%%%%%%%%%%%%%%%%%
			\begin{question}{/}{Dérivées}{1}{22 et 23 et 24}
				Une population de bactéries croit en fonction du temps avec la relation suivante : $P(t) = \frac{4t}{t^{2}+1}$. Calculer la dérivée ce cette fonction en fonction du temps, c'est-à-dire calculer la vitesse de croissance de cette population de bactéries.
            \end{question}
            \begin{reponses}
            	\item[false] $\frac{4(1+3t^{2})}{(t^{2}+1)^{2}}$
            	\item[false]  $ 4(1-t^{2})$
                \item[true]  $\frac{4(1-t^{2})}{(t^{2}+1)^{2}}$
                \item[false] $ \frac{4(1-t^{2})}{(2t+1)}$
            \end{reponses}
			%%%%%%%%%%%%%%%%%%%%%%%%%%%%%%%%%%%%%
