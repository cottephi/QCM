         \begin{question}{22}{Dérivées}{3}{/}
               Soit f une fonction définie sur $\mathbb{R}^{3}$ telle que $f(x,y,z) = zcos(xy)$. Que vaut la dérivée partielle de f par rapport à x, notée $\frac{\partial f}{\partial x}$ ?
        \end{question}
        \begin{reponses}
        	\item[false]  $z\sin(xy)$
        	\item[false]  $z\sin(y)+zcos(xy)$
            \item[false]  $\frac{z}{\cos(xy)^2}$
            \item[true]  $-zy\sin(xy)$
        \end{reponses}
         \begin{question}{22}{Dérivées}{3}{/}
               Soit f une fonction définie sur $\mathbb{R}^{2}$ telle que $f(x,y) = cos(xcos(y))$. Que vaut la dérivée partielle de f par rapport à y, notée $\frac{\partial f}{\partial y}$ ?
        \end{question}
        \begin{reponses}
        	\item[true]  $x\sin(y)\sin(x\cos(y))$
        	\item[false]  $-x\sin(y)\sin(x\cos(y))$
            \item[false]  $\sin(y)\cos(x\cos(y))$
            \item[true]  $-\sin(y)\cos(x\cos(y))$
        \end{reponses}
        %%%%%%%%%%%%%%%%%%%%%%%%%%%%%%%%%%%%%%%
          \begin{question}{NC}{dérivées partielles}{2}{/} 
            La dérivée partielle d'une fonction, à deux variables, par rapport à une variable est sa dérivée par rapport à cette variable en considérant l'autre variable comme constante. Soit f une fonction définie sur $\mathbb{R}^{2}$ telle que $f(x,y) = x + 2y $. Donner la dérivée partielle de f par rapport à y notée $\frac{\partial f}{\partial y}$.  
            \end{question}
            \begin{reponses}
            	\item[true]  $2$
            	\item[false]  $x+2$
                \item[false]   $2y$
                \item[false]    $x$
            \end{reponses}
			%%%%%%%%%%%%%%%%%%%%%%%%%%%%%%%%%%%%%
        \begin{question}{22}{Dérivées}{3}{/}
               Soit f une fonction définie sur $\mathbb{R}^{3}$ telle que $f(x,y,z) = z\frac{x^{3}+y}{x^{2}+xy}$. Que vaut la dérivée partielle de f par rapport à z, notée $\frac{\partial f}{\partial z}$ ?
        \end{question}
        \begin{reponses}
        	\item[false]  $z$
        	\item[false]  $-z\frac{3x^{2}+1}{(x^{2}+y)^2}$
            \item[true]  $\frac{x^{3}+y}{x^{2}+xy}$
            \item[false]  $z+\frac{x^{3}+y}{x^{2}+xy}$
        \end{reponses}
        %%%%%%%%%%%%%%%%%%%%%%%%%%%%%%%%%%%%%%%
            \begin{question}{NC}{dérivées partielles}{2}{/} 
            La dérivée partielle d'une fonction, à deux variables, par rapport à une variable est sa dérivée par rapport à cette variable en considérant l'autre variable comme constante. Soit f une fonction définie sur $\mathbb{R}^{2}$ telle que $f(x,y) = \exp(10xy)+x\cos(5y)$. Donner la dérivée partielle de f par rapport à y notée $\frac{\partial f}{\partial y}$.  
            \end{question}
            \begin{reponses}
            	\item[false]  $\frac{\partial f}{\partial y} = 10x\exp(10xy)$
            	\item[false]  $\frac{\partial f}{\partial y} =10x\exp(10xy)+5x\sin(5y)$
                \item[false]  $\frac{\partial f}{\partial y} =-5x\sin(5y)$
                \item[true]   $\frac{\partial f}{\partial y} =10x\exp(10xy)-5x\sin(5y)$
            \end{reponses}
			%%%%%%%%%%%%%%%%%%%%%%%%%%%%%%%%%%%%%
