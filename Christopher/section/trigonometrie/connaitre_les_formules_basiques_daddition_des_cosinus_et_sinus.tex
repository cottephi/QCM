            \begin{question}{47}{Trigonométrie}{1}{/}
				Soient $\alpha$ et $\beta$ deux angles quelconques. Quelle est l'expression de $\sin(\alpha+\beta)$?
            \end{question}
            \begin{reponses}
            	\item[false] $\cos(\alpha)\cos(\beta)+\sin(\alpha)\sin(\beta)$
            	\item[true] $\sin(\alpha)\cos(\beta)+\cos(\alpha)\sin(\beta)$
                \item[false] $\cos(\alpha)\cos(\beta)-\sin(\alpha)\sin(\beta)$
                \item[false] $\sin(\alpha)\cos(\beta)-\sin(\alpha)\sin(\beta)$
            \end{reponses}
			%%%%%%%%%%%%%%%%%%%%%%%%%%%%%%%%%%%%%
			\begin{question}{47}{Trigonométrie}{1}{/}
				Soient $p et $q deux angles quelconques. Quelle est l'expression de $\cos(p)+\cos(q)$?
            \end{question}
            \begin{reponses}
            	\item[true] $\cos(p) + \cos(q) = 2\cos(\frac{p+q}{2}).\cos(\frac{p-q}{2})$
            	\item[false] $\cos(p) - \cos(q) = -2\sin(\frac{p+q}{2}).\sin(\frac{p-q}{2})$
                \item[false] $\sin(p) + \sin(q) = 2\sin(\frac{p+q}{2}).\cos(\frac{p-q}{2})$
                \item[false] $\sin(p) - \sin(q) = 2\cos(\frac{p+q}{2}).\sin(\frac{p-q}{2}) $
            \end{reponses}
			%%%%%%%%%%%%%%%%%%%%%%%%%%%%%%%%%%%%%
