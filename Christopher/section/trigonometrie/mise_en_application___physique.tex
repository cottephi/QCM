            \begin{question}{47}{Trigonométrie}{2}{/}
				Soient deux fonctions $f(t) = \cos(2\pi f t)$ et $g(t) = \cos(2\pi f t + \phi)$ périodiques dans le temps $t$. La fréquence temporelle est notée $f$ et  $\phi$ représente le décalage de phase entre les deux fonctions, c'est-à-dire le fait qu'elles vibrent à la même fréquence mais décalées dans le temps. Donner l'expression de $f(t) + g(t)$
            \end{question}
            \begin{reponses}
            	\item[false] $2\cos(\frac{\phi}{2})\cos(2\pi ft -\frac{\phi}{2})$
            	\item[false] $2\sin(\frac{\phi}{2})\cos(2\pi ft +\frac{\phi}{2})$
                \item[true] $2\cos(\frac{\phi}{2})\cos(2\pi ft +\frac{\phi}{2})$
                \item[false] $2\cos(\frac{\phi}{2})\sin(2\pi ft -\frac{\phi}{2})$
            \end{reponses}
			%%%%%%%%%%%%%%%%%%%%%%%%%%%%%%%%%%%%%
        	\begin{question}{47}{Trigonométrie}{3}{/}
				La lumière est une onde électromagnétique qui peut se propager dans le vide. Comme toute onde, elle possède une fréquence temporelle, notée $f$, et une période spatiale, notée $\lambda$. On peut décrire mathématique la propagation d'une onde progressive (qui se propage dans le temps et dans l'espace) par la formule suivante : $W(x,t) = A\cos(kx - \omega t)$, avec $k=\frac{2\pi}{\lambda}$ et $\omega = 2\pi f$. Si l'on fait se rencontrer deux ondes de même fréquence $f$, il se produit ce qu'on appelle des interférences. Soient deux ondes progressives $W_1(x,t) = A\cos(kx - \omega t)$ et $W_2(x,t) = A\cos(kx - \omega t + \phi)$. $\phi$ représente le décalage de phase entre les deux ondes, c'est-à-dire le fait qu'elles vibrent à la même fréquence mais décalées dans le temps. Donner l'expression de $W_1(x,t) + W_2(x,t)$.
            \end{question}
            \begin{reponses}
            	\item[true] $2A\cos(\frac{\phi}{2})\cos(kx -wt +\frac{\phi}{2})$
            	\item[false] $2A\sin(\frac{\phi}{2})\cos(kx -wt +\frac{\phi}{2})$
                \item[false] $2A\cos(\frac{\phi}{2})\cos(kx -wt -\frac{\phi}{2})$
                \item[false] $2A\cos(\frac{\phi}{2})\sin(kx -wt -\frac{\phi}{2})$
            \end{reponses}
			%%%%%%%%%%%%%%%%%%%%%%%%%%%%%%%%%%%%%
