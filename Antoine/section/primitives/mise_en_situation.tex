        % "CH156 Thermochimie
        % calculer une enthalpie de réaction
        % calculer un travail"
		\begin{question}{60,61}{Enthalpie}{2}{}
			La formule générale de l'enthalpie d'une réaction chimique à pression et température constante est $\displaystyle \Delta H = \int^{x_f}_{0}\Delta_r H(x)\, \mathrm{d}x$, elle se mesure en Joule. $\Delta_r H$ est l'enthalpie due à l'avancement de la réaction chimique. Ici $x_f$ désigne l'avancement final du système en moles. Admettons que \emph{trois moles} de la réaction \ce{C12H22O11 -> 12 C + 11H2O} ont lieu et que $\Delta_rH(x) = x$. Quelle est l'enthalpie de la réaction?
		\end{question}
		\begin{reponses}
			\item[false] $\Delta H = \SI{5/2}{\joule}$
			\item[false] $\Delta H = \SI{3/2}{\joule}$
			\item[true] $\Delta H = \SI{4.5}{\joule}$
			\item[false] $\Delta H = \SI{3}{\joule}$
		\end{reponses}
		%%%%%%%%%%%%%%%%%%%%
		\begin{question}{60,61}{Enthalpie}{2}{}
			La formule générale de l'enthalpie d'une réaction chimique à pression et température constante est $\displaystyle \Delta H = \int^{x_f}_{0}\Delta_r H(x)\, \mathrm{d}x$, elle se mesure en Joule. $\Delta_r H$ est l'enthalpie due à l'avancement de la réaction chimique. Ici $x_f$ désigne l'avancement final du système en moles. Admettons que \emph{dix moles} de la réaction \ce{C12H22O11 -> 12 C + 11H2O} ont lieu et que $\Delta_rH(x) = 3$. Quelle est l'enthalpie de la réaction?
		\end{question}
		\begin{reponses}
			\item[false] $\Delta H = 27x\,\si{\joule}$
			\item[true] $\Delta H = \SI{30}{\joule}$
			\item[false] $\Delta H = \SI{4}{\joule}$
			\item[false] $\Delta H = \SI{27}{\joule}$
		\end{reponses}
		%%%%%%%%%%%%%%%%%%%%
		\begin{question}{60,61}{Enthalpie}{2}{}
			La formule générale de l'enthalpie d'une réaction chimique à pression et température constante est $\displaystyle \Delta H = \int^{x_f}_{0}\Delta_r H(x)\, \mathrm{d}x$, elle se mesure en Joule. $\Delta_r H$ est l'enthalpie due à l'avancement de la réaction chimique. Ici $x_f$ désigne l'avancement final du système en moles. Admettons que \emph{deux moles} ($x_f = \SI{2}{\mole}$) de la réaction \ce{CH4 + 2O2 -> CO2 + 2H2O} ont lieu et que $\Delta_rH(x) = x^2$. Quelle est l'enthalpie de la réaction?
		\end{question}
		\begin{reponses}
			\item[false] $\Delta H = \SI{11/3}{\joule}$
			\item[false] $\Delta H = \SI{14/5}{\joule}$
			\item[false] $\Delta H = \SI{1/2}{\joule}$
			\item[true] $\Delta H = \SI{8/3}{\joule}$
		\end{reponses}
		%%%%%%%%%%%%%%%%%%%%
		\begin{question}{60,61}{Enthalpie}{2}{}
			La formule générale de l'enthalpie d'une réaction chimique à pression constante est $\displaystyle \Delta H = C_P^{sys} \Delta T + \int^{x_f}_{0}\Delta_r H(x)\, \mathrm{d}x$, elle se mesure en Joule. $C_P^{sys}$ est la capacité calorifique à pression constante du système. $\Delta T$ est la variation de température du système au cours de la réaction, $\Delta_r H$ est l'enthalpie due à l'avancement de la réaction chimique. Ici $x_f$ désigne l'avancement final du système en moles. Admettons que \emph{deux moles} ($x_f = \SI{2}{\mole}$) de la réaction \ce{CH4 + 2O2 -> CO2 + 2H2O} ont lieu et que $\Delta_rH(x) = \frac{4}{5}x^3$. La réaction se fait à température constante, quelle est l'enthalpie de la réaction?
		\end{question}
		\begin{reponses}
			\item[false] $\Delta H = \SI{11/5}{\joule}$
			\item[false] $\Delta H = \SI{14/5}{\joule}$
			\item[false] $\Delta H = \SI{1/2}{\joule}$
			\item[true] $\Delta H = \SI{16/5}{\joule}$
		\end{reponses}
		%%%%%%%%%%%%%%%%%%%%
		\begin{question}{60,61,1214}{Enthalpie}{2}{}
			La formule générale de l'enthalpie d'une réaction chimique à pression constante est $\displaystyle \Delta H = C_P^{sys} \Delta T + \int^{x_f}_{0}\Delta_r H(x)\, \mathrm{d}x$, elle se mesure en Joule. $C_P^{sys}$ est la capacité calorifique à pression constante du système. $\Delta T$ est la variation de température du système au cours de la réaction, $\Delta_r H$ est l'enthalpie due à l'avancement de la réaction chimique. Ici $x_f$ désigne l'avancement final du système en moles. Admettons que \emph{quatre moles} ($x_f = \SI{4}{\mole}$) de la réaction \ce{2H + O2 -> 2H2O} ont lieu et que $\Delta_rH(x) = 4e^{3x}$. La réaction se fait à température constante, quelle est l'enthalpie de la réaction?
		\end{question}
		\begin{reponses}
			\item[false] $\Delta H \simeq \SI{217}{\joule}$
			\item[true] $\Delta H \simeq \SI{217}{\kilo\joule}$
			\item[false] $\Delta H \simeq \SI{.5}{\milli\joule}$
			\item[false] $\Delta H \simeq \SI{27}{\joule}$
		\end{reponses}
		%%%%%%%%%%%%%%%%%%%%
		\begin{question}{60,61,1214}{Enthalpie}{3}{}
			À température et pression constante, l'entropie d'une réaction chimique est donnée par la formule $\displaystyle \Delta S[\si{\joule\per\kelvin}] = \int^{x_f}_{0}\Delta_r S(x)\, \mathrm{d}x$. $\Delta_r S$ est l'entropie instantanée de la réaction chimique. Ici $x_f$ désigne l'avancement final du système en mole. Quelle est l'expression générale de l'entropie sachant que l'entropie instantanée de la réaction étudiée est $\Delta_r S(x) = 3e^{10x}-12x^2+5-7x$?
		\end{question}
		\begin{reponses}
			\item[false] $\displaystyle\Delta S = -\frac{10 {x_f}^{4/3}}{\sqrt{3}}-\frac{7 {x_f}^2}{2}+\frac{3 e^{10 {x_f}}}{10}+7$
			\item[false] $\displaystyle\Delta S = -\frac{4 {x_f}^{3/2}}{\sqrt{3}}-\frac{7 {x_f}^2}{2}+\frac{3 e^{10 {x_f}}}{10}$
			\item[false] $\displaystyle\Delta S = 0$
			\item[true] $\displaystyle\Delta S = 5 x_f - (7 x_f^2)/2 - 4 x_f^3 + \frac{3 (-1 + e^{10 {x_f}})}/{10}$
		\end{reponses}
		%%%%%%%%%%%%%%%%%%%%
