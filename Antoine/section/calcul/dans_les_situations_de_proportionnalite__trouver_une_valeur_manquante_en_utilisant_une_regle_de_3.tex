		\subsubsection{Dilution}
			\begin{question}{}{Dilution}{1}{}
				Lors d'une dilution, quelle est la formule liant la concentration des solutions mère et fille ($C_{m,f}$) aux volumes de dilution et prélevé de la solution mère ($V_{f,m}$)?
			\end{question}
			\begin{reponses}
				\item[true] $C_f\cdot V_f = C_m\cdot V_m$
				\item[false] $C_f\cdot V_m = C_m\cdot V_f$
				\item[false] $C_f/V_f = C_m/V_m$
				\item[false] $V_f/C_f = V_m/C_m$
			\end{reponses}
			%%%%%%%%%%%%%%%%%%%%
			\begin{question}{1215,1209,1210,1211}{Dilution}{2}{}
				Lors d'une dilution, les concentrations et volumes des solutions mères et filles sont reliés par $C_f\cdot V_f = C_m\cdot V_m$. Quelle est la concentration d'une solution de la solution fille après dilution par un facteur 100 (c'est à dire que $V_f/V_m = 100$) de la solution mère de HCl à $C_m = \SI{5e2}{\mole\per\liter}$?
			\end{question}
			\begin{reponses}
				\item[false] $C_f = \SI{500}{\mole\per\liter}$
				\item[false] $C_f = \SI{5e4}{\mole\per\liter}$
				\item[true] $C_f = \SI{5}{\mole\per\liter}$
				\item[true] $C_f = \SI{5e3}{\milli\mole\per\liter}$
			\end{reponses}
			%%%%%%%%%%%%%%%%%%%%
			\begin{question}{1215,1209}{Dilution}{2}{}
				Lors d'une dilution, les concentrations et volumes des solutions mères et filles sont reliés par $C_f[\si{\mole\per\liter}]\cdot V_f[\si{\liter}] = C_m[\si{\mole\per\liter}]\cdot V_m[\si{\liter}]$. On souhaite diluer une solution d'acide fluorhydrique à $C_m = \SI{.5}{\mole\per\liter}$ pour obtenir une solution \emph{trois fois moins concentrée}. Quel est le facteur de dilution (ratio $V_f/V_m$) que l'on doit appliquer pour réaliser cette opération?
			\end{question}
			\begin{reponses}
				\item[true] $V_f/V_m = \num{3}$
				\item[false] $V_f/V_m = \num{0.3}$
				\item[false] $V_f/V_m = 1/3$
				\item[false] $V_f/V_m = \SI{3}{\liter}$
			\end{reponses}
			%%%%%%%%%%%%%%%%%%%%
			\begin{question}{1215,1209,1214,1210,1211}{dilution}{3}{}
				Lors d'une dilution, les concentrations et volumes des solutions mères et filles sont reliés par $C_f\cdot V_f = C_m\cdot V_m$. La législation en matière de stockage du mystérieux agent X a changé, la concentration maximale autorisée dans une solution est désormais de \SI{1}{\mol\per\liter} contre \SI{1}{\kilo\mol\per\liter} auparavant. On dispose d'un volume de \SI{1}{\meter\cubed} qu'il nous faut diluer au plus vite sous peines de poursuites judiciaires. Quelle quantité de liquide obtiendrons-nous après cette dilution? Cela rentre-t-il dans les \emph{quatre cuves de \SI{18000}{\liter}} dont nous disposons?
			\end{question}
			\begin{reponses} 
				\item[true] \SI{1e3}{\meter\cubed}
				\item[false] \SI{10}{\meter\cubed}
				\item[true] Non.
				\item[false] Oui.
			\end{reponses}
			%%%%%%%%%%%%%%%%%%%%
			\begin{question}{1215,1209,1214,1210,1211}{Dilution}{3}{}
				Lors d'une dilution, les concentrations et volumes des solutions mères et filles sont reliés par $C_f\cdot V_f = C_m\cdot V_m$. On souhaite diluer une solution d'acide ascorbique à $C_m = \SI{5e2}{\milli\mole\per\liter}$ pour obtenir une solution \emph{dix fois moins concentrée}. Cependant, le récipient dont on dispose pour effectuer cette dilution n'est que de $\SI{50}{\centi\liter}$. Quel volume de la solution initiale doit-on prélever? On ne prend pas en compte le fait que le récipient ne peut se remplir à raz bord en pratique et on considère avoir assez de solution mère à disposition.
			\end{question}
			\begin{reponses} 
				\item[true] \SI{5}{\centi\liter}
				\item[false] On ne peut pas.
				\item[false] \SI{5}{\centi\meter\cubed}
				\item[false] \SI{5}{\milli\liter}
			\end{reponses}
			%%%%%%%%%%%%%%%%%%%%
			\begin{question}{1215,1209,1214,1210,1211}{Dilution}{3}{}
				Lors d'une dilution, les concentrations et volumes des solutions mères et filles sont reliés par $C_f\cdot V_f = C_m\cdot V_m$. On souhaite diluer une solution d'acide ascorbique à $C_m = \SI{5e2}{\milli\mole\per\liter}$ pour obtenir une solution dont la concentration est de $C_f = \SI{3}{\milli\mole\per\liter}$. Cependant, le récipient dont on dispose pour effectuer cette dilution n'est que de $\SI{50}{\centi\liter}$. Quel volume de la solution initiale doit-on prélever? On ne prend pas en compte le fait que le récipient ne peut se remplir à raz bord en pratique et on considère avoir assez de solution mère à disposition.
			\end{question}
			\begin{reponses} 
				\item[false] \SI{5}{\micro\liter}
				\item[false] On ne peut pas.
				\item[true] \SI{3}{\centi\meter\cubed}
				\item[true] \SI{3}{\milli\liter}
			\end{reponses}
			%%%%%%%%%%%%%%%%%%%%
			\begin{question}{1215,1209,1214,1210,1211}{Dilution}{3}{}
			Un cafetier italien immigré aux États-Unis doit adapter sa boisson au goût local. Les \textit{ristretto} italiens sont typiquement d'un volume de \SI{2}{\centi\liter}. Les consommateurs américains, quant-à eux apprécient les cafés plus allongés servis dans un gobelet en carton de \SI{50}{\centi\liter}. Ces mêmes clients apprécient également un café moins fort, c'est pourquoi notre cafetier décide d'utiliser la même quantité de café pour les deux préparations (on peut donc considérer que la quantité de matière dissoute dans les deux boissons est la même). Quelle est alors la concentration relative de l'\textit{americano} par rapport à celle du \textit{ristretto}, $C_a/C_r$?
			\end{question}
			\begin{reponses} 
				\item[true] \SI{4}{\percent}
				\item[false] \SI{0.04}{\percent}
				\item[false] \SI{2500}{\percent}
					\item[false] \SI{100}{\percent}
			\end{reponses}
			%%%%%%%%%%%%%%%%%%%%
			\begin{question}{1215,1209,1214,1210,1211}{Homéopathie 1}{3}{}
				L'Oscillococcinum, \frquotes{médicament} homéopathique des laboratoires Boiron est fabriqué à partir d'extraits de foie et de c\oe{}ur de canard de Barbarie à \SI{200}{K}. Un \si{K} pour \textit{dynamisation de Korsakov}, correspond à \textit{grosso modo} une dilution au centième de la préparation originale suivie d'une étape appelée dynamisation (agitation de la solution). Quel est le ratio de la concentration mère par rapport à la solution fille après l'élaboration du soit disant médicament?
			\end{question}
			\begin{reponses} 
				\item[true] $C_f/C_m = \num{e-400}$
				\item[false] $C_f/C_m = \num{e200}$
				\item[false] $C_f/C_m = \num{e-100}$
				\item[false] $C_f/C_m = \num{e-200}$
			\end{reponses}
			%%%%%%%%%%%%%%%%%%%%
			\begin{question}{1215,1209,1214,1210,1211}{Homéopathie 2}{3}{}
				La solution constitué des abats de canard servant à la préparation d'Oscillococcinum est au départ concentrée à \SI{50}{\gram\per\liter} (\SI{35}{\gram} de foie et \SI{15}{\gram} de c\oe{}ur pour une préparation de \SI{1}{\liter}). Quelle est alors la masse d'abats de canard contenue dans \emph{une unique bille} de lactose (le médicament final) sachant qu'on dispose sur l'ensemble des granules d'un tube (soit 200 granules de \SI{5}{\milli\gram}), \SI{0.01}{\milli\liter} de la solution diluée à \SI{200}{K} (ce qui correspond à $C_f/C_m = \num{e-400}$). À titre informatif, la masse d'un atome de carbone, principal constituant de la matière \frquotes{vivante}, est \SI{2e-26}{\kilo\gram} environ.
			\end{question}
			\begin{reponses} 
				\item[true] \SI{2.5e-406}{\gram\per granule}
				\item[false] \SI{2.6e-7}{\gram\per granule}
				\item[false] \SI{5e-399}{\gram\per granule}
				\item[false] \SI{5e-404}{\gram\per granule}
			\end{reponses}
			%%%%%%%%%%%%%%%%%%%%
