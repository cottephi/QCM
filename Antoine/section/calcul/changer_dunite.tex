		\begin{question}{1215,1214}{Conversions}{1}{}
            L'électronvolt est l'unité commune d'énergie utilisé en physique nucléaire et en atomistique. On définit un électron-volt comme l'énergie cinétique acquise par un électron dans un potentiel de un volt après avoir parcouru \SI{1}{\meter}, soit environ \SI{1.602e-19}{\joule}. La première raie (H$_\alpha$) de l'hydrogène est de longueur d'onde \SI{656.2}{\nano\meter}, ce qui correspond à une énergie de \SI{3.03e-19}{\joule}. À quoi cela correspond-t-il en électronvolts?
        \end{question}
        \begin{reponses}
		    \item[false] Environ \SI{4.856e-38}{\electronvolt}.
		    \item[false] \SI{-13.6}{\electronvolt}
		    \item[true] Environ \SI{1.89}{\electronvolt}.
		    \item[false] Environ \SI{3}{\electronvolt}.
        \end{reponses}
        %%%%%%%%%%%%%%%%%%%%
		\begin{question}{}{Conversions}{1}{}
            À combien de \si{\meter} correspond \SI{1}{\angstrom} (\frquotes{un Ångström})?
        \end{question}
        \begin{reponses}
            \item[false] \SI{1e-9}{\meter}
            \item[true] \SI{1e-10}{\meter}
            \item[false] \SI{1e10}{\meter}
            \item[false] \SI{1e-3}{\meter}
        \end{reponses}
        %%%%%%%%%%%%%%%%%%%%
		\begin{question}{}{Conversions}{1}{}
			À quoi correspond un \si{ppm} (partie par million)?
        \end{question}
        \begin{reponses} 
            \item[true] \SI{1e-4}{\percent}
            \item[false] \SI{0.01}{\percent}
            \item[true] \num{1e-6}
    	    \item[false] \SI{100}{\percent}
        \end{reponses}
        %%%%%%%%%%%%%%%%%%%%
		\begin{question}{1215,1214}{Conversions}{1}{}
            À combien de \si{\milli\liter} correspond \SI{1}{\centi\meter\cubed}?
        \end{question}
        \begin{reponses}
            \item[false] \SI{10}{\milli\liter}
            \item[false] \SI{1000}{\milli\liter}
            \item[true] \SI{1}{\milli\liter}
            \item[false] \SI{100}{\milli\liter}
        \end{reponses}
        %%%%%%%%%%%%%%%%%%%%
        \begin{question}{1215,1214}{Conversions}{1}{}
            À combien de \si{\liter} correspond \SI{1}{\meter\cubed}?
        \end{question}
        \begin{reponses}
            \item[false] \SI{1}{\liter}
            \item[true] \SI{1000}{\liter}
            \item[false] \SI{10}{\liter}
            \item[false] \SI{100}{\liter}
        \end{reponses}
        %%%%%%%%%%%%%%%%%%%%
        \begin{question}{}{Conversions}{1}{}
            Quelle est la formule générale de conversion des Kelvin en degrés Celsius?
        \end{question}
        \begin{reponses}
            \item[true] $T[\si{\celsius}] = T[\si{\kelvin}] - \num{273.15}$
            \item[false] $T[\si{\celsius}] = T[\si{\kelvin}] + \num{273.15}$
            \item[false] $T[\si{\kelvin}] = T[\si{\celsius}] - \num{273.15}$
            \item[false] $T[\si{\kelvin}] = T[\si{\celsius}] + \num{273.15}$
        \end{reponses}
        %%%%%%%%%%%%%%%%%%%%
        \begin{question}{}{Conversions}{1}{}
            Quelle est la formule générale de conversion des degrés Celsius en Kelvin?
        \end{question}
        \begin{reponses}
            \item[false] $T[\si{\celsius}] = T[\si{\kelvin}] - \num{273.15}$
            \item[false] $T[\si{\celsius}] = T[\si{\kelvin}] + \num{273.15}$
            \item[false] $T[\si{\kelvin}] = T[\si{\celsius}] - \num{273.15}$
            \item[true] $T[\si{\kelvin}] = T[\si{\celsius}] + \num{273.15}$
        \end{reponses}
        %%%%%%%%%%%%%%%%%%%%
        \begin{question}{}{Conversions}{2}{}
            Quelles sont les affirmations suivantes vraies?
        \end{question}
        \begin{reponses}
            \item[true] $\SI{1e-9}{\meter} = \SI{1}{\nano\meter}$
            \item[true] $\SI{10e-10}{\meter} = \SI{10}{\angstrom}$
            \item[false] $\SI{1e-3}{\meter} = \SI{3}{\angstrom}$
            \item[false] $\SI{1e-3}{\joule} = \SI{3}{\angstrom}$
        \end{reponses}
        %%%%%%%%%%%%%%%%%%%%
		\begin{question}{}{Conversions}{2}{}
            Quelles sont les affirmations suivantes vraies?
        \end{question}
        \begin{reponses}
            \item[true] $\SI{1e-2}{\mole\per\centi\meter\cubed} = \SI{10}{\mole\per\liter}$
            \item[false] $\SI{1}{\milli\mole\per\centi\meter\cubed} = \SI{10}{\mole\per\liter}$
            \item[true] $\SI{1e4}{\mole\per\liter} = \SI{10}{\mole\per\centi\meter\cubed}$
            \item[true] $\SI{0.1}{\milli\mole\per\centi\meter\cubed} = \SI{0.1}{\mole\per\liter}$
        \end{reponses}
        %%%%%%%%%%%%%%%%%%%%
        \begin{question}{}{Conversions}{2}{}
            Quelles sont les affirmations suivantes vraies?
        \end{question}
        \begin{reponses}
            \item[true] $\SI{1e-2}{\percent} = \num{1e-4}$
            \item[false] Une atténuation de $3/4$ équivaut à une atténuation de \num{.8}.
            \item[true] $\SI{18}{\percent} = \num{0.18}$
            \item[false] $\SI{18}{\percent} = \num{1.8}$
        \end{reponses}
        %%%%%%%%%%%%%%%%%%%%
         \begin{question}{}{Conversions}{2}{}
            Quelles sont les affirmations suivantes vraies?
        \end{question}
        \begin{reponses}
            \item[false] $\SI{2}{\percent} = \num{0.2}$
            \item[true] Une atténuation de $3/4$ équivaut à une amplification de \num{-.75}.
            \item[false] $\SI{3}{\milli\liter} = \num{3e4}$
            \item[true] $\SI{6}{\meter\squared} = \SI{60000}{\centi\meter\squared}$
        \end{reponses}
        %%%%%%%%%%%%%%%%%%%%
        \begin{question}{}{Conversions}{2}{}
            Quelles sont les affirmations suivantes vraies?
        \end{question}
        \begin{reponses}
            \item[false] $\SI{0.7}{\percent} = \num{0.07}$
            \item[false] Une atténuation de $4/8$ équivaut à une atténuation de \num{-0.5}.
            \item[true] $\SI{3}{\milli\liter} = \SI{3e3}{\milli\meter\cubed}$
            \item[true] $\SI{6}{\meter\squared} = \SI{6e-6}{\kilo\meter\squared}$
        \end{reponses}
        %%%%%%%%%%%%%%%%%%%%
        \begin{question}{}{Conversions}{3}{}
            Quelles sont les affirmations suivantes vraies? On rappelle que $\mathcal{N_A} = \SI{6.02e23}{\per\mole}$.
        \end{question}
        \begin{reponses}
            \item[true] Trois tours par minute équivalent à \SI{0.314}{\radian\per\second}.
            \item[true] $\SI{1}{\mole\per\liter} = \SI{1}{\milli\mole\per\centi\meter\cubed}$
            \item[false] Pour un élément de masse molaire $M = \SI{5}{\gram\per\mole}$: $\SI{3}{\mole\per\liter} = \SI{15}{\gram\per\meter\cubed}$.
            \item[false] $\SI{50}{\kilo\mole\per\meter\cubed} \SI{3e20}{\milli\mole\per\centi\meter\cubed}$
        \end{reponses}
        %%%%%%%%%%%%%%%%%%%%
		\begin{question}{}{Conversions}{3}{}
            Quelles sont les affirmations suivantes vraies?
        \end{question}
        \begin{reponses}
            \item[true] $\SI{1e-9}{\meter\cubed} = \SI{1}{\micro\liter}$
            \item[false] $\SI{3e-30}{\liter} = \SI{3}{\angstrom\cubed}$
            \item[true] $\SI{10}{\liter} = \SI{1e-3}{\meter\cubed}$
            \item[false] $\SI{1e-3}{\meter\cubed} = \SI{3}{\angstrom}$
        \end{reponses}
        %%%%%%%%%%%%%%%%%%%%
        \begin{question}{}{Conversions}{3}{}
            Quelles sont les affirmations suivantes vraies?
        \end{question}
        \begin{reponses}
            \item[true] $\SI{70}{\kelvin} = \SI{-203.15}{\celsius}$
            \item[false] $\SI{373.15}{\celsius} = \SI{100}{\kelvin}$
            \item[true] $\SI{4}{\kelvin\squared} \simeq \SI{7.24e4}{\celsius\squared}$
            \item[true] $\SI{300}{\kelvin} \simeq \SI{20}{\celsius}$
        \end{reponses}
        %%%%%%%%%%%%%%%%%%%%
