		\begin{question}{}{Formules}{1}{}
			Comment calculer l'erreur sur la mesure de l'énergie d'un photon $E=\hbar\omega$? On notera l'erreur sur la mesure de la pulsation du photon $\delta\omega$ et l'erreur sur la constante de Planck réduite $\delta\hbar$.
        \end{question}
        \begin{reponses}
    	    \item[false] $\delta E = \abs{\frac{\partial E}{\partial \hbar}}\delta \hbar+\abs{\frac{\partial E}{\partial \omega}}\delta \omega$
		    \item[true] $\delta E = \sqrt{\abs{\frac{\partial E}{\partial \hbar}\delta \hbar}^2+\abs{\frac{\partial E}{\partial \omega}\delta \omega}^2}$
		    \item[false] $\delta E = \sqrt{\abs{\frac{\partial E}{\partial \hbar}\delta \hbar}^2-\abs{\frac{\partial E}{\partial \omega}\delta \omega}^2}$
		    \item[false] $\delta E = \sqrt{\frac{\partial E}{\partial \hbar}\delta \hbar^2+\frac{\partial E}{\partial \omega}\delta \omega^2}$
        \end{reponses}
        %%%%%%%%%%%%%%%%%%%%
        \begin{question}{1209}{Formules}{2}{}
			Une définition simplifiée du pH est: $\text{pH} = -\log_{10}\left(\text{[H\textsuperscript{+}]}\right)$, avec la concentration donnée en \si{\mol\per\liter}. Quelle est la concentration en ion H\textsuperscript{+} d'une solution de pH de 8? Est-elle acide ou basique?
        \end{question}
        \begin{reponses}
    	    \item[false] $\text{[H\textsuperscript{+}]} = \SI{1.13}{\mol\per\liter}$ et la solution est basique.
    	    \item[true] $\text{[H\textsuperscript{+}]} = \SI{1.0e-8}{\mol\per\liter}$ et la solution est basique.
    	    \item[false] $\text{[H\textsuperscript{+}]} = \SI{1.0e-8}{\mol\per\liter}$ et la solution est acide.
    	    \item[false] $\text{[H\textsuperscript{+}]} = \SI{2}{\mol\per\liter}$ et la solution est acide.
        \end{reponses}
        %%%%%%%%%%%%%%%%%%%%
        \begin{question}{1209}{Formules}{2}{}
			Une définition simplifiée du pH est: $\text{pH} = -\log_{10}\left(\text{[H\textsuperscript{+}]}\right)$, avec la concentration donnée en \si{\mol\per\liter}. Quelle est la concentration en ion H\textsuperscript{+} d'une solution de pH de 7? Est-elle acide ou basique?
        \end{question}
        \begin{reponses}
    	    \item[false] $\text{[H\textsuperscript{+}]} = \SI{2}{\mol\per\liter}$ et la solution est basique.
    	    \item[false] $\text{[H\textsuperscript{+}]} = \SI{e-7}{\mol\per\liter}$ et la solution est acide.
    	    \item[true] $\text{[H\textsuperscript{+}]} = \SI{0.1}{\micro\mol\per\liter}$ et la solution est neutre.
    	    \item[false] $\text{[H\textsuperscript{+}]} = \SI{2}{\mol\per\liter}$ et la solution est acide.
        \end{reponses}
        %%%%%%%%%%%%%%%%%%%%
        \begin{question}{1209,1210}{pH}{2}{}
			On donne les constantes $K_1 = 1000$, $K_2 = \num{1e14}$ et $K_3 = K_1/K_2$. On définit le pK comme $\text{pK} = -\log_{10}\left(K\right)$. Que vaut pK\textsubscript{3}?
        \end{question}
        \begin{reponses}
    	    \item[true] $\text{pK}_3 = 11$
    	    \item[false] $\text{pK}_3 = 1000$
    	    \item[false] $\text{pK}_3 = \num{1e-11}$
    	    \item[false] $\text{pK}_3 = 0$
        \end{reponses}
        %%%%%%%%%%%%%%%%%%%%
        \begin{question}{1209,1210}{pH}{2}{}
			On donne les constantes $K_1 = 1000$, $K_2 = 3$ et $K_3 = K_1^{K_2}$. On définit le pK comme $\text{pK} = -\log_{10}\left(K\right)$. Que vaut pK\textsubscript{3}?
        \end{question}
        \begin{reponses}
    	    \item[false] $\text{pK}_3 = 10$    
    	    \item[true] $\text{pK}_3 = -9$
    	    \item[false] $\text{pK}_3 = \num{1e9}$
    	    \item[false] $\text{pK}_3 = 0$
        \end{reponses}
        %%%%%%%%%%%%%%%%%%%%
        \begin{question}{1209,1210}{pH}{2}{}
			On donne les constantes $K_1 = 1000$, $K_2 = 3$ et $K_3 = K_1^{-K_2}$. On définit le pK comme $\text{pK} = -\log_{10}\left(K\right)$. Que vaut pK\textsubscript{3}?
        \end{question}
        \begin{reponses}
    	    \item[false] $\text{pK}_3 = 5$    
    	    \item[false] $\text{pK}_3 = -9$
    	    \item[false] $\text{pK}_3 = \num{1e-9}$
    	    \item[false] $\text{pK}_3 = 0$
        \end{reponses}
        %%%%%%%%%%%%%%%%%%%%
        \begin{question}{1210,1211}{Formules}{2}{}
			Le moment dipolaire d'un système de deux charges est le produit de la distance entre les deux charge multiplié par la valeur de la charge positive du système. Les atomes d'hydrogène et de fluor de l'acide fluorhydrique sont séparés d'une distance de \SI{0.92}{\angstrom}. Sachant que la charge positive du système est de $e\,\si{\coulomb}$, quel est le \emph{moment dipolaire} du système?
        \end{question}
        \begin{reponses}
    	    \item[true] $\mu = \SI{1.47e-29}{\coulomb\meter}$
    	    \item[false] $\mu = \SI{0.92}{\coulomb\meter}$
    	    \item[false] $\mu = \SI{1.47e-28}{\coulomb\meter}$
    	    \item[false] $\mu = \SI{0.92e-30}{\coulomb\meter}$
        \end{reponses}
        %%%%%%%%%%%%%%%%%%%%
		\begin{question}{1209}{Formules}{2}{}
			Une définition simplifiée du pH est: $\text{pH} = -\log_{10}\left(\text{[H\textsuperscript{+}]}\right)$, avec la concentration donnée en \si{\mol\per\liter}. Quel est le pH d'une solution dont la concentration en ion H\textsuperscript{+} est de \SI{0.1}{\milli\mol\per\liter}? Est-elle acide ou basique?
        \end{question}
        \begin{reponses}
    	    \item[false] Le pH est de \num{4} et la solution est basique.
    	    \item[true] Le pH est d'environ \num{4} et la solution est acide.
    	    \item[false] Le pH est d'environ \num{7} et la solution est neutre.
    	    \item[false] Le pH est d'environ \num{7} et la solution est basique.
        \end{reponses}
        %%%%%%%%%%%%%%%%%%%%
		\begin{question}{1209}{Formules}{2}{}
            Une définition simplifiée du pH est: $\text{pH} = -\log_{10}\left(\text{[H\textsuperscript{+}]}\right)$, avec la concentration donnée en \si{\mol\per\liter}. Quel est le pH d'une solution dont la concentration en ion H\textsuperscript{+} est de \SI{0.1}{\mol\per\liter}? Est-elle acide ou basique?
        \end{question}
        \begin{reponses}
    	    \item[false] Le pH est de \num{1.69} et la solution est basique.
    	    \item[true] Le pH est d'environ \num{1.7} et la solution est acide.
    	    \item[false] Le pH est d'environ \num{7} et la solution est neutre.
    	    \item[false] Le pH est d'environ \num{9} et la solution est basique.
        \end{reponses}
        %%%%%%%%%%%%%%%%%%%%
		\begin{question}{1209}{Formules}{2}{}
            Une définition simplifiée du pH est: $\text{pH} = -\log_{10}\left(\text{[H\textsuperscript{+}]}\right)$, avec la concentration donnée en \si{\mol\per\liter}. Quelle est la concentration en ion H\textsuperscript{+} d'une solution de pH de -9? Est-elle acide ou basique?
        \end{question}
        \begin{reponses}
    	    \item[false] $\text{[H\textsuperscript{+}]} = \SI{9}{\mol\per\liter}$ et la solution est basique.
    	    \item[false] On ne peut pas savoir.
    	    \item[false] $\text{[H\textsuperscript{+}]} = \SI{1}{\giga\mol\per\liter}$ et la solution est basique.
    	    \item[false] $\text{[H\textsuperscript{+}]} = \SI{1e-9}{\mol\per\liter}$ et la solution est neutre.
        \end{reponses}
        %%%%%%%%%%%%%%%%%%%%
		\begin{question}{1209}{Formules}{2}{}
            Une définition simplifiée du pH est: $\text{pH} = -\log_{10}\left(\text{[H\textsuperscript{+}]}\right)$, avec la concentration donnée en \si{\mol\per\liter}. Quelle est la concentration en ion H\textsuperscript{+} d'une solution de pH de 9? Est-elle acide ou basique?
        \end{question}
        \begin{reponses}
    	    \item[false] $\text{[H\textsuperscript{+}]} = \SI{9}{\mol\per\liter}$ et la solution est basique.
    	    \item[true] $\text{[H\textsuperscript{+}]} = \SI{1e-9}{\mol\per\liter}$ et la solution est basique.
    	    \item[true] $\text{[H\textsuperscript{+}]} = \SI{1}{\nano\mol\per\liter}$ et la solution est basique.
    	    \item[false] $\text{[H\textsuperscript{+}]} = \SI{2}{\mol\per\liter}$ et la solution est neutre.
        \end{reponses}
        %%%%%%%%%%%%%%%%%%%%
        \begin{question}{1209}{Formules}{2}{}
			Une définition simplifiée du pH est: $\text{pH} = -\log_{10}\left(\text{[H\textsuperscript{+}]}\right)$, avec la concentration donnée en \si{\mol\per\liter}. Quelle est la concentration en ion H\textsuperscript{+} d'une solution de pH de 20? Est-elle acide ou basique?
        \end{question}
        \begin{reponses}
    	    \item[false] $\text{[H\textsuperscript{+}]} = \SI{9}{\mol\per\liter}$ et la solution est basique.
    	    \item[false] $\text{[H\textsuperscript{+}]} = \SI{1e-20}{\mol\per\liter}$ et la solution est basique.
    	    \item[false] $\text{[H\textsuperscript{+}]} = \SI{1}{\nano\mol\per\liter}$ et la solution est basique.
    	    \item[false] $\text{[H\textsuperscript{+}]} = \SI{2}{\mol\per\liter}$ et la solution est neutre.
        \end{reponses}
        %%%%%%%%%%%%%%%%%%%%
        \begin{question}{1209,1210}{pH}{2}{}
			On donne les constantes $K_1 = 1000$, $K_2 = \num{1e-2}$ et $K_3 = K_1\times K_2$. On définit le pK comme $\text{pK} = -\log_{10}\left(K\right)$. Que vaut pK\textsubscript{3}?
        \end{question}
        \begin{reponses}
    	    \item[false] $\text{pK}_3 = 100$
    	    \item[false] $\text{pK}_3 = \num{-1e-4}$
    	    \item[true] $\text{pK}_3 = -1$
    	    \item[false] $\text{pK}_3 = 0$
        \end{reponses}
        %%%%%%%%%%%%%%%%%%%%
		\begin{question}{1210,1211}{Formules}{2}{}
            L'énergie $E$ d'un photon est donnée par la relation: $E=hc/\lambda$, où $c$ est la célérité de la lumière en \si{\meter\per\second}, $\lambda$ est la longueur d'onde en \si{\meter} et $h$ est la constante de Planck en \si{\joule\second}. Dans quel cas la relation $E=h\nu$ est également juste?
        \end{question}
        \begin{reponses}
            \item[false] $\nu$ est la vitesse du son en \si{\meter\per\second}.
            \item[false] $\nu$ est la longueur d'onde en \si{\nano\meter}.
            \item[true] $\nu$ est la fréquence de la lumière en \si{\per\second}.
            \item[false] $\nu$ est la période de la lumière en \si{\second}.
        \end{reponses}
        %%%%%%%%%%%%%%%%%%%%
		\begin{question}{1210,1214,1215}{Formules}{2}{}
			Le Debye noté \si{D} est une unité de mesure du moment dipolaire, $\SI{1}{D} = \SI{3.34e-30}{\coulomb\meter}$. Quelles sont les conversions ci-après justes?
        \end{question}
        \begin{reponses}
    	    \item[true] $\SI{3}{D} \simeq  \SI{1.0e-29}{\coulomb\meter}$
    	    \item[true] $\SI{9e29}{D} \simeq  \SI{3}{\coulomb\meter}$
    	    \item[true] $\SI{1e-4}{D} \simeq  \SI{3.3e-34}{\coulomb\meter}$
    	    \item[true] $\SI{6e14}{D} \simeq  \SI{2.0e-19}{\coulomb\meter}$
        \end{reponses}
        %%%%%%%%%%%%%%%%%%%%
		\begin{question}{1210,1211}{Formules}{3}{}
			Les atomes d'hydrogène et de fluor de l'acide fluorhydrique sont séparés d'une distance de \SI{0.92}{\angstrom}. Sachant que la charge positive du système est de $e\,\si{\coulomb}$, quel est le \emph{moment dipolaire} du système?
        \end{question}
        \begin{reponses}
    	    \item[true] $\mu = \SI{1.47e-29}{\coulomb\meter}$
    	    \item[false] $\mu = \SI{0.92}{\coulomb\meter}$
    	    \item[false] $\mu = \SI{1.47e-28}{\coulomb\meter}$
    	    \item[false] $\mu = \SI{0.92e-30}{\coulomb\meter}$
        \end{reponses}
        %%%%%%%%%%%%%%%%%%%%
        \begin{question}{1210,1211}{Formules}{3}{}
            Les atomes d'hydrogène et de chlore de l'acide chlorhydrique sont séparés d'une distance de \SI{1.27e-10}{\meter}. Sachant que la charge positive du système est de $e\,\si{\coulomb}$ (la charge de l'électron), quel est le \emph{moment dipolaire} du système?
        \end{question}
        \begin{reponses}
    	    \item[true] $\mu = \SI{2.03e-29}{\coulomb\meter}$
    	    \item[false] $\mu = \SI{1.27}{\coulomb\meter}$
    	    \item[false] $\mu = \SI{2.03e-28}{\coulomb\meter}$
    	    \item[false] $\mu = \SI{1.27e-30}{\coulomb\meter}$
        \end{reponses}
        %%%%%%%%%%%%%%%%%%%%
        \begin{question}{1210,1211}{Formules}{3}{}
            Les atomes d'hydrogène et de brome du bromure d'hydrogène sont séparés d'une distance de \SI{0.141}{\nano\meter}. Sachant que la charge positive du système est de $e\,\si{\coulomb}$ (la charge de l'électron), quel est le \emph{moment dipolaire} du système?
        \end{question}
        \begin{reponses}
    	    \item[true] $\mu = \SI{2.26e-29}{\coulomb\meter}$
    	    \item[false] $\mu = \SI{1.41}{\coulomb\meter}$
    	    \item[false] $\mu = \SI{1.58e-28}{\coulomb\meter}$
    	    \item[false] $\mu = \SI{3.40e-30}{\coulomb\meter}$
        \end{reponses}
        %%%%%%%%%%%%%%%%%%%%
        \begin{question}{1210,1211}{Formules}{3}{}
			Les atomes d'hydrogène et de chlore de la molécule de iodure d'hydrogène sont séparés d'une distance de \SI{1.61e-10}{\meter}. Sachant que la charge positive du système est de $e\,\si{\coulomb}$ (la charge de l'électron), quel est le \emph{moment dipolaire} du système?
        \end{question}
        \begin{reponses}
    	    \item[true] $\mu = \SI{2.57e-29}{\coulomb\meter}$
    	    \item[false] $\mu = \SI{1.61}{\coulomb\meter}$
    	    \item[false] $\mu = \SI{2.03e-28}{\coulomb\meter}$
    	    \item[false] $\mu = \SI{1.27e-30}{\coulomb\meter}$
        \end{reponses}
        %%%%%%%%%%%%%%%%%%%%
