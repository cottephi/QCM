		\begin{question}{8}{Équations}{2}{}
            Quelles sont les valeurs des coefficients de st\oe{}chiométrie $a$ et $b$ de la réaction chimique \ce{CH4 + a\,O2 -> CO2 + b\,H2O}?
        \end{question}
        \begin{reponses}
            \item[false] $a=1$ et $b=2$.
            \item[false] $a=2$ et $b=1$.
            \item[false] $a=4$ et $b=3$.
            \item[true] $a=2$ et $b=2$.
        \end{reponses}
        %%%%%%%%%%%%%%%%%%%%
        \begin{question}{8}{Équations}{2}{}
            Quelles sont les valeurs des coefficients de st\oe{}chiométrie $a$, $b$ et $c$ de la réaction chimique \ce{a\,H2 + b\,O2 -> c\,H2O}?
        \end{question}
        \begin{reponses}
            \item[false] $a=1$, $b=2$ et $c=1$.
            \item[true] $a=2$, $b=1$ et $c=2$.
            \item[false] $a=4$, $b=0$ et $c=3$.
            \item[false] $a=2$, $b=2$ et $c=2$.
        \end{reponses}
        %%%%%%%%%%%%%%%%%%%%
		\begin{question}{8}{Équations}{2}{}
			Quelles sont les valeurs des coefficients de st\oe{}chiométrie $a$, $b$ et $c$ de la réaction chimique \ce{a\,N2 + b\,H2 -> c\,NH3}?
		\end{question}
		\begin{reponses}
			\item[false] $a=0$, $b=3$ et $c=1$.
			\item[true] $a=1$, $b=3$ et $c=2$.
			\item[false] $a=4$, $b=3$ et $c=3$.
			\item[false] $a=2$, $b=2$ et $c=2$.
		\end{reponses}
		%%%%%%%%%%%%%%%%%%%%
		\begin{question}{1216,8,1209}{Calcul de concentration}{2}{}
            L'air est composé majoritairement de di-azote (\emph{deux} atomes d'azote), de di-oxygène (\emph{deux} atomes d'oxygène) et d'\emph{un pourcent} de gaz rares (majoritairement de l'argon). La masse molaire de l'air est de \SI{28.965}{\gram\per\mol}. Quelle est la proportion des différentes espèces dans ce mélange, sachant que la masse molaire de l'azote est \SI{14.007}{\gram\per\mol}, celle de l'oxygène est \SI{15.999}{\gram\per\mol} et celle de l'argon \SI{39.948}{\gram\per\mol}? Rappelez-vous que la somme des proportions des différentes espèces est de \SI{100}{\percent}, c'est à dire que $1 = x_{\ce{N}} + x_{\ce{O}} + x_{\ce{Ar}}$ où $x_i$ désigne la proportion de l'élément en question.
        \end{question}
        \begin{reponses}
            \item[false] \SI{80.112}{\percent} d'azote, \SI{19.365}{\percent} d'oxygène et \SI{0.952}{\percent} d'argon.
		    \item[true] \SI{78.125}{\percent} d'azote, \SI{20.875}{\percent} d'oxygène et \SI{1}{\percent} d'argon.
		    \item[false] On ne peut pas le calculer.
		    \item[false] \SI{79.103}{\percent} d'azote, \SI{19.897}{\percent} d'oxygène et \SI{1}{\percent} d'argon.
        \end{reponses}
        %%%%%%%%%%%%%%%%%%%%
		\begin{question}{1216,8,1209}{Calcul de concentration}{3}{}
            L'air est composé majoritairement de di-azote, de di-oxygène et d'\emph{un pourcent} de gaz rares (majoritairement de l'argon). La masse molaire de l'air est de \SI{28.965}{\gram\per\mol}. Quelle est la proportion des différentes espèces dans ce mélange, sachant que la masse molaire de l'azote est \SI{14.007}{\gram\per\mol}, celle de l'oxygène est \SI{15.999}{\gram\per\mol} et celle de l'argon \SI{39.948}{\gram\per\mol}?
        \end{question}
        \begin{reponses}
            \item[false] \SI{80.112}{\percent} d'azote, \SI{19.365}{\percent} d'oxygène et \SI{0.952}{\percent} d'argon.
		    \item[true] \SI{78.125}{\percent} d'azote, \SI{20.875}{\percent} d'oxygène et \SI{1}{\percent} d'argon.
		    \item[false] On ne peut pas le calculer.
		    \item[false] \SI{79.103}{\percent} d'azote, \SI{19.897}{\percent} d'oxygène et \SI{1}{\percent} d'argon.
        \end{reponses}
        %%%%%%%%%%%%%%%%%%%%
		\begin{question}{1216,8,1209}{Calcul de concentration}{3}{}
            L'air est composé majoritairement de \SI{78.125}{\percent} de di-azote, de di-oxygène et d'\emph{un pourcent} de gaz rares (majoritairement de l'argon). La masse molaire de ces composants est respectivement de \SI{14.007}{\gram\per\mol}, \SI{15.999}{\gram\per\mol} et \SI{39.948}{\gram\per\mol}. Quelle est la masse molaire de l'air?
        \end{question}
        \begin{reponses}
            \item[false] \SI{40.140}{\gram\per\mol}
		    \item[true] \SI{28.965}{\gram\per\mol}
		    \item[false] On ne peut pas la calculer.
		    \item[false] \SI{34.605}{\gram\per\mol}
        \end{reponses}
        %%%%%%%%%%%%%%%%%%%
